\documentclass[conference]{IEEEtran}
\IEEEoverridecommandlockouts
% The preceding line is only needed to identify funding in the first footnote. If that is unneeded, please comment it out.
\usepackage{cite}
\usepackage{amsmath,amssymb,amsfonts}
\usepackage{algorithmic}
\usepackage{graphicx}
\usepackage{textcomp}
\usepackage{xcolor}
\usepackage{caption}
\usepackage{subcaption}
\usepackage{lscape}
\usepackage{longtable}
\usepackage{csvsimple}
\usepackage{booktabs}
\usepackage{etoolbox}
\usepackage{adjustbox}
\usepackage{array}
\usepackage{xcolor,colortbl}
\usepackage{hyperref}
\usepackage{siunitx}
\usepackage{import}
\usepackage{natbib}
\usepackage{numprint}
\npthousandsep{\,}
\usepackage{xstring}

\def\BibTeX{{\rm B\kern-.05em{\sc i\kern-.025em b}\kern-.08em
    T\kern-.1667em\lower.7ex\hbox{E}\kern-.125emX}}

% pragma once, adapted from https://tex.stackexchange.com/a/195173
\makeatletter
\let\pragma@iinput=\@iinput
\def\@iinput#1{\xdef\@pragmafile{#1}\pragma@iinput{#1}}
\def\@pragmafile{default}
\def\pragmaonce{%
   \csname pragma@\@pragmafile\endcsname
   \global\expandafter\let \csname pragma@\@pragmafile\endcsname = \endinput
}
\makeatother

\ifdefined\mydraft
\mydraft
\fi

\begin{document}

\title{Conference Paper Title*\\
{\footnotesize \textsuperscript{*}Note: Sub-titles are not captured in Xplore and
should not be used}
\thanks{Identify applicable funding agency here. If none, delete this.}
}

\author{\IEEEauthorblockN{Matthew Andres Moreno}
\IEEEauthorblockA{\textit{Computer Science and Engineering} \\
\textit{Michigan State University}\\
East Lansing, United States \\
\url{https://orcid.org/0000-0003-4726-4479}
}
\and
\IEEEauthorblockN{Charles Ofria}
\IEEEauthorblockA{\textit{Computer Science and Engineering} \\
\textit{Michigan State University}\\
East Lansing, United States \\
\url{https://orcid.org/0000-0003-2924-1732}
}
}

\maketitle

\begin{IEEEkeywords}
component, formatting, style, styling, insert
\end{IEEEkeywords}

\begin{abstract}
This document is a model and instructions for \LaTeX.
This and the IEEEtran.cls file define the components of your paper [title, text, heads, etc.]. *CRITICAL: Do Not Use Symbols, Special Characters, Footnotes,
or Math in Paper Title or Abstract.
\end{abstract}


\subsection{Modeling Modalities and the Nature of Artificial Life}

The roots of scientific understanding lie in juxtaposition of abstract models with the natural world \citep{banzhaf2016defining}.
Modality and structure of these models vary tremendously, but all relate axiomatic formulations to consequent implications.
Verbal models relate abstractions through qualitative logic.
Mathematical models interface quantified entities through geometric, algebraic, analytic, combinatorial, and other formal devices.
Crucially, symbolic representations allow mathematical approaches to resolve sophisticated relationships through compact, closed-form descriptions.
Finally, physical models extend back into the natural world, constructing tangible analogs, generally as footholds for exploration of verbal and mathematical models (i.e., down-scaled or stripped down).

Digital electronic processing ushered in computational models, largely unrealized up to that point.
Such models reify vast layers --- today, trillions --- of exactly prescribed interactions.
This capability grants entirely new avenues to study collective behavior of complex systems.
Although alike in telling outcomes consequent to axiomatic suppositions, the scope of some computational models, however, can erode their descriptiveness in elucidating intermediate mechanistic steps.
Akin to physical models, in some cases it becomes necessary to resort to further cycles of experimentation and model-building analogous to those applied against the natural world.
Artificial life, largely coming to fruition at the advent of modern computing, rests at this juncture.
The field employs models, physical and computational, as an experimental subject to scaffold understanding through verbal and mathematical models \citep{bedau2003artificial}.

Today, the divide between physical and computational modeling modalities is generally hard-cut.
Unlike physical models, computational outcomes are perfectly attributable: mathematical specification wholly determines every state change.
Although exact in tracing out abstraction, computation occurs through physical processes --- in the case of digital electronic computing, via voltage and current in physical hardware.
Borrowing terminology typically applied to situating of cognition \citep{etxeberria1998embodiment}, computation in practice is physically embodied.
Usually, this fact holds relevant only insofar as performance considerations structure particulars in program design.
Recently, however, concern has grown over mounting costs of preserving perfect rectitude across ever more expansive contingents of hardware.

In this piece, we discuss alternative ``best-effort'' strategies, starting first with a critical examination of their relationship to artificial life ethos and existing research agendas within the field.
We then propose a pragmatic approach to make better use of existing high-performance computing (HPC) hardware, with promising initial results in exploratory benchmarks.

\subsection{Best-effort Computing}

As HPC scales, it becomes increasingly difficult to write software that makes efficient use of available hardware and, simultaneously, provides reproducible results (or even near-perfectly reproducible results --- i.e., up to effects from floating point non-transitivity).
Difficulty largely boils down to two quagmires:
\begin{enumerate}
\item synchronicity paces to the slowest laggard, which tends towards increasing extremity as pool size grows; and
\item incidence of rare compound failures rises with component count, necessitating greater redundancies to prevent fatal error.
\end{enumerate}
Although subject of many fruitful mitigation efforts, costs nonetheless mount in performance overhead and engineering effort.
More insidious, still, are soft errors --- faults, such as lost messages or corrupted memory, that transpire silently \citep{karnik2004characterization}.
Further miniaturization and voltage reduction, which are assumed as a likely vehicle for continuing advances in hardware efficiency and performance, could conceivably worsen susceptibility to such errors \citep{dongarra2014applied,kajmakovic2020challenges}.
Vanishingly rare at the scale of single-processor computing and generally mitigable through redundancy (e.g., duplicate execution or error correction codes \citep{vankeirsbilck2015soft,sridharan2015memory}), countermeasures must ramp up with scale as occurrence accelerates to non-trivial rates \citep{sridharan2015memory,scoles2018cosmic}.

The worsening marginal costs of synchronization, fault recovery, and error correction beg the question whether it is viable to maintain, or even strive to maintain, the reliable digital machine model at scale \citep{dongarra2014applied}.
Indeed, software and hardware that concede some incorrectness or indeterminism --- the ``best-effort model'' --- have been shown to improve speed \citep{chakrapani2008probabilistic}, energy efficiency \citep{chakrapani2008probabilistic,bocquet2018memory}, and scalability \citep{meng2009best}.
Discussion around ``approximate computing'' overlaps significantly with ``best-effort computing,'' although focusing more heavily on using algorithm design to sidestep non-essential computation (i.e., reduced floating point precision, inexact memoization, etc.) \citep{mittal2016survey}.
However, best-effort computing stands distinct in breach of execution's adherenece to idealization.
The intrusion of extraneous influence into programmed execution relaxes best-effort simulation into a sort of liminally physical model.
In this light, the best-effort proposition appears starkly exotic.
In an alternate framing, however, this step simply represents further pragmatic compromise with the physical reality of computational hardware --- with the difference of kind manifested as spillover from constraining aspects of program specification to eroding the specifiable purview.

The suitability of best-effort approaches varies from application to application.
Real-time control systems that cannot afford to pause or retry, by necessity, fall into the best-effort category \citep{rahmati2011computing, rhodes2020real}.
Other domains are clear cut against departure from the reliable digital machine model --- for example, due to regulatory issues \citep{dongarra2014applied}.
However, a subset of HPC applications can tolerate occasional flaws, or even fundamentally nondeterminism, in computation \citep{chakradhar2010best}.
Optimization domains, typically having no singular ``correct'' outcome, make an obvious fit;
those, especially, leveraging heuristics or pseudo-stochastic methods that tend to exploit noise rather than destabilize due to it \citep{chakrapani2008probabilistic,chakradhar2010best}.
In this vein, best-effort applications to stochastic gradient descent for artificial neural network applications have proven fruitful \citep{dean2012large,zhao2019elastic,niu2011hogwild,noel2014dogwild,rhodes2020real}.
Even within directly representational simulation, however, notable success has stemmed from machine learning approximations of simulation mechanics \citep{behler2007generalized,kochkov2021machine}.
(Noted, such work categorizes as ``approximate'' rather than ``best-effort.'')

The pliability of artificial life behooves best-effort strategy.
As artificial life already brings together purely-computational, explicitly-physical (i.e., robotics) and purely-physical (i.e., biochemical) modeling modalities, the blurred nature of best-effort computational models make a natural fit.
Indeed, best-effort, real-time computing approaches have already made several notable appearances in artificial life work.
Tom Ray's Network Tierra pushed cyber-physical model hybridization to the logical extreme.
This project instantiated populations across an at-will federation of distributed simulation endpoints, with individuals migrating across peer-to-peer internet connections.
Exposure of simulation dynamics to external factors was, in fact, an explicit aim rather than an engineering compromise;
influence of human-driven fluctuations in internet traffic and server load due to seasonality, day/night cycles, etc. on evolved behaviors was highlighted of key interest \citep{ray1995proposal}.

Most notable in systematically advocating, developing, and realizing best-effort simulation, though, is David Ackley's work on indefinite scalability in the context of open-ended evolution.
This line of work paves a trajectory toward clustering modular distributed hardware at a theoretically unlimited scale.
Best-effort asynchronicity and error-tolerance come as necessary preconditions to this end, as does decentrality and spatiality \citep{ackley2011pursue}.
Notable accomplishments in this vein include the Movable Feast Machine (MFM) execution framework, designed as a substrate for distributed spatial automata \citep{ackley2013movable}; the ULAM programming language, designed for expressive MFM configurations \citep{ackley2016ulam}; and, more recently, impressive emergent structures and replicators engineered within within these substrates \citep{ackley2018digital,ackley2023robust}.
Ackley's work is specially notable in its extension to custom hardware demonstrative of indefinite scalability principles, including Illuminata X Machina \citep{ackley2011homeostatic} and, presently, the T2 Tile \citep{ackley2023robust}.
Perhaps the most emphasized thread through Ackley's oeuvre, though, is a broad wholehearted boosterism for the significance of substrate capacity to artificial life's core ambitions \citep{ackley2014indefinitely}.
Suggestions to this end appear elsewhere \citep{channon2019maximum,banzhaf2016defining,moreno2019toward}, as well, and we echo emphasis on system scale-up here.

\subsection{Scale in Context}

A reference point is instructive to discussing scope in artificial life simulation.
Take as an example the Avida digital evolution platform, a popular software system for evolutionary experiments with self-replicating computer programs.
In this system, a population of ten thousand digital organisms can undergo approximately twenty thousand generations per day \citep{ofria2009artificial}.
This equates to about two hundred million individual replication cycles.
Each flask in the Lenski Long-Term Evolution Experiment hosts a similar number of replication cycles; with an effective population size of 30 million \textit{E. coli} that undergo a bit more than 6.6 doublings per day, the bacteria experience about 180 million replication events per day \citep{good2017dynamics}.
As another point of commparison, in Ratcliff’s work studying the evolution of multicellularity in \textit{S. cerevisiae}, about six doublings per day occur among a population numbering on the order of a billion cells \citep{ratcliff2012experimental}.

Although colloquialised as ``worlds,'' with serial processing power the scope of typical artificial life simulation aligns, in naive terms, near a laboratory flask.
Of course, such a comparison neglects profound incommensurabilities between Avidians and bacteria or yeast.
Natural organisms have vastly more genetic content and phenotypic state, as well as more and more diverse interactions with the environment and with other cells.

Since the field's inception, artificial life has ridden a steady current of microprocessor innovation.
As improvements in serial processing dried up around the turn of the century \citep{sutter2005free}, parallel and distributed became ascendant.
As a result, it has become routine to dispatch independent instantiations of simulation runs across hardware units.
In scientific contexts, this practice yields replicate datasets that provide statistical power to answer research questions \citep{dolson2017spatial}.
In applied contexts, this practice yields many converged populations that can be scavenged for the best solutions overall \citep{hornby2006automated}.
Another established practice is to use ``island models'' where individuals are transplanted between populations residing across distributed hardware \citep{gorges1990explicit}.
Notably, asynchronous approaches are common and effective in these models \citep{abdelhafez2019performance}.

Koza and collaborators’ genetic programming work with a 1,000-CPU Beowulf cluster typifies the island model approach \citep{bennett1999building}.
In recent years, Sentient Technologies spearheaded evolutionary computation projects on an unprecedented computational scale, comprising over a million CPUs and capable of a peak performance of 9 petaflops \citep{miikkulainen2019evolving}.
According to its proponents, the scale and scalability of this ``DarkCycle'' system was a key aspect of its conceptualization \citep{gilbert2015artificial}.
Much of the assembled infrastructure was pieced together from heterogeneous providers and employed on a time-available basis \citep{blondeau2009distributed}.
Unlike typical island models where selection occurs entirely independently on each CPU, this scheme transferred evaluation criteria between computational instances in addition to individual genomes \citep{hodjat2013distributed}.
Sentient Technologies also notably exploited a large pool of hardware accelerators (e.g., 100 GPUs) in work evolving neural network architectures by performing each candidate architecture's costly model training and evaluation process \citep{miikkulainen2019evolving}.

Existing parallel and distributed digital evolution systems typically minimize interaction between simulation components on disjoint hardware.
Such independence facilitates simple and efficient implementation.
This approach typically involves independent evaluation of sub-populations (i.e., island models) or individuals (i.e., primary-subordinate or controller-responder parallelism \citep{cantu2001master}).
Cases where evaluation of a single individual are parallelized often involve data-parallel evaluation over a set of independent test cases, which are subsequently consolidated into a single fitness profile \citep{harding2007fast_springer, langdon2019continuous}.

However, several notable parallel and distributed digital evolution systems have incorporated rich interactions between parallelized simulation components.
Harding applied GPU acceleration to cellular automata models of artificial development systems, which involve intensive interaction between spatially-distributed instantiation of a genetic program \citep{harding2007fast_ieee}.
Likewise, Network Tierra featured arbitrary communication between digital organisms residing on different machines \citep{ray1995proposal}.
More recently, in a continuation of much earlier work, Christian Heinemann's ongoing ALIEN project has leveraged GPU acceleration for perform physics-based simulation of soft body agents within a 2D arena \citep{heinemann2008artificial}.

\section{Paths Forward}

Implicit among much of the field, it seems, is an anticipation that order-of-magnitude changes in artificial life systems may unlock a qualitative sea change in simulation outcomes.
This is not an entirely unreasonable notion of a possible future of artificial life.

Spectacular advances achieved with artificial neural networks over the last decade illuminate one conceivable progression of events.
As with digital evolution, artificial neural networks (ANNs) were traditionally understood as a versatile but auxiliary methodology --- both techniques have been described as ``the second best way to do almost anything'' \citep{miaoulis2008intelligent,eiben2015introduction}.
However, the utility and ubiquity of ANNs has since increased dramatically \citep{marcus2018deep}.
The development of AlexNet is widely considered pivotal to this transformation.
AlexNet united methodological innovations from the field (such as big datasets, dropout, and ReLU) with GPU computing that enabled training of orders-of-magnitude-larger networks.
In fact, some aspects of their deep learning architecture were expressly modified to accommodate multi-GPU training \citep{krizhevsky2012imagenet}.
By adapting existing methodology to exploit commercially available hardware, AlexNet spurred the greater availability of compute resources to the research domain and eventually the introduction of custom hardware to expressly support deep learning \citep{jouppi2017datacenter}.

Likewise, progress of artificial life systems along a path, in the most ambitious framing, to indefinite scalability seems likely to unfold through incremental investments spurred by progressive scientific achievements.
To that end, observability \citep{moreno2023toward} and interpretability \citep{horgan1995complexity} will be key concerns, as will intentionally focuusing engineering effort to support hypothesis-driven objectives.
In tandem with more farsighted efforts, progress will also require prosaic leverage of existing, commercially-available parallel and distributed compute resources at circumstantially-feasible scales.

Such work should be made with an eye for contribution back to HPC.
As noted earlier, the unique character of artificial life simulation suits it to serve at the tip of the spear in HPC evolution.
High-performance computing hardware with transformative capabilities is coming to market right now, and some of it --- like the Cerebras wafer scale engine \citep{lauterbach2021path} --- is built explicitly for decentralized, asynchronous computation.
Given ubiquity of deep net training and stencil-based numerical solvers in applications (CITE), programming for agent-based simulation can be of interest insofar as it flexes harware capabilities in unimagined ways.
Effort to establish artificial life simulation as a flagship HPC application could be of mutual benefit.

\subsection{On-hardware Experiments}

Here, we present a set of on-hardware experiments to empirically characterize Conduit's best-effort communication model.
In order to survey across workload profiles, we tested performance under both a communication-intensive graph coloring solver and a compute-intensive artificial life simulation.

First, we determine whether best-effort communication strategies can benefit performance compared to the traditional perfect communication model.
We considered two measures of performance: computational steps executed per unit time and solution quality achieved within a fixed-duration run window.

We compare the best-effort and perfect-computation strategies across processor counts, expecting to see the marginal benefit from best-effort communication increase at higher processor counts.
We focus on weak scaling, growing overall problem size proportional to processor count.
Put another way, we hold problem size per processor constant.%
\footnote{
As opposed to strong scaling, where the problem size is held fixed while processor count increases.
}
This approach prevents interference from shifts in processes' workload profiles in observation of the effects of scaling up processor count.

To survey across hardware configurations, we tested scaling CPU count via threading on a single node and scaling CPU count via multiprocessing with each process assigned to a distinct node.
In addition to a fully best-effort mode and a perfect communication mode, we also tested two intermediate, partially synchronized modes: one where the processor pool completed a global barrier (i.e., they aligned at a synchronization point) at predetermined, rigidly scheduled time points and another where global barriers occurred on a rolling basis spaced out by fixed-length delays from the end of the last synchronization.%
\footnote{
Our motivation for these intermediate synchronization modes was interest in the effect of clearing any potentially-unbounded accumulation of message backlogs on laggard processes.
}

Second, we sought to more closely characterize variability in message dispatch, transmission, and delivery under the best-effort model.
Unlike perfect communication, real-time volatility affects the outcome of computation under the best-effort model.
Because real-time processing speed degradations and message latency or loss alters inputs to simulation units, characterizing the distribution of these phenomena across processing components and over time is critical to understanding the actual computation being performed.
For example, consistently faster execution or lower messaging latency for some subset of processing elements could violate uniformity or symmetry assumptions within a simulation.
It is even possible to imagine reciprocal interactions between real-time best-effort dynamics and simulation state.
In the case of a positive feedback loop, the magnitude of effects might become extreme.
For example, in artificial life scenarios, agents may evolve strategies that selectively increase messaging traffic so as to encumber neighboring processing elements or even cause important messages to be dropped.

We monitor five aspects of real-time behavior, which we refer to as quality of service metrics \citep{karakus2017quality},
\begin{itemize}
  \item wall-time simulation update rate (``update period''),
  \item simulation-time message latency,
  \item wall-time message latency,
  \item steadiness of message inflow (``delivery bunching''), and
  \item delivery failure rate.
\end{itemize}

In an initial set of experiments, we use the graph coloring problem to test this suite of quality of service metrics across runtime conditions expected to strongly influence them.
We compare
\begin{itemize}
  \item increasing compute workload per simulation update step,
  \item within-node versus between-node process placement, and
  \item multithreading versus multiprocessing.
\end{itemize}
We perform these experiments using a graph coloring solver configured to maximize communication relative to computation (i.e., just one simulation unit per CPU) in order to maximize sensitivity of quality of service to the runtime manipulations.

Finally, we extend our understanding of performance scaling from the preceding experiments by analyzing how each quality of service metric fares as problem size and processor count grow together, a ``weak scaling'' experiment.
This analysis would detect a scenario where raw performance remains stable under weak scaling, but quality of service (and, therefore, potentially quality of computation) degrades.

Experiments employ a new software framework, the Conduit C++ Library for Best-Effort High Performance Computing \citep{moreno2021conduit}.
The Conduit library provides tools to perform best-effort communication in a flexible, intuitive interface and uniform inter-operation of serial, parallel, and distributed modalities.
Although Conduit currently implements distributed functionality via MPI intrinsics, future work will explore lower-level protocols like InfiniBand Unreliable Datagrams \citep{kashyap2006ip, koop2007high}.


% be sure to give example of how it could play out under different asynchronicity modes @MAM

\section{Methods}

We performed two benchmarks to compare the performance of Conduit's best-effort approach to a traditional synchronous model.
We tested our benchmarks across both a multithread, shared-memory context and a distributed, multinode context.
In each hardware context, we assessed performance on two algorithmic contexts: a communication-intensive distributed graph coloring problem (Section \ref{sec:graph-coloring-benchmark}) and a compute-intensive digital evolution simulation (Section \ref{sec:digital_evolution_benchmark}).
The latter benchmark --- presented in Section \ref{sec:digital_evolution_benchmark} --- grew out of the original work developing the Conduit library to support large-scale experimental systems to study open-ended evolution.
The former benchmark --- presented in Section \ref{sec:graph-coloring-benchmark} --- complements the first by providing a clear definition of solution quality.
Metrics to define solution quality in the open-ended digital evolution context remain a topic of active research.

\subsection{ Digital Evolution Benchmark } \label{sec:digital_evolution_benchmark}

The digital evolution benchmark runs the DISHTINY (DIStributed Hierarchical Transitions in Individuality) artificial life framework.
This system is designed to study major transitions in evolution, events where lower-level organisms unite to form a self-replicating entity.
The evolution of multicellularity and eusociality exemplify such transitions.
Previous work with DISHTINY has explored methods for selecting traits characteristic of multicellularity such as reproductive division of labor, resource sharing within kin groups, resource investment in offspring, and adaptive apoptosis \citep{moreno2019toward}.

DISHTINY simulates a fixed-size toroidal grid populated by digital cells.
Cells can sense attributes of their immediate neighbors, can communicate with those neighbors through arbitrary message passing, and can interact with neighboring cells cooperatively through resource sharing or competitively through antagonistic competition to spawn daughter cells into limited space.
This cell behavior is controlled by SignalGP event-driven linear genetic programs \citep{lalejini2018evolving}.
Full details of the DISHTINY simulation are available in \citep{moreno2021exploring}.

We use Conduit-based messaging channels to manage all interactions between neighboring cells.
Conduit models messaging channels as independent objects.
However, support is provided for behind-the-scenes consolidation of communication along these channels between pairs of processes.
Pooling joins together exactly one message per messaging channel to create a fixed-size consolidated message.
Aggregation joins together arbitrarily many messages per channel  to  create a variable-size consolidated message.

During a computational update, each cell advances its internal state and pushes information about its current state to neighbor cells.
Several independent messaging layers handle disparate aspects of cell-cell interaction, including
\begin{itemize}
  \item Cell spawn messages, which contain arbitrary-length genomes (seeded at 100 12-byte instructions with a hard cap of 1000 instructions). These are handled every 16 updates and use Conduit's built-in aggregation support for inter-process transfer.
  \item Resource transfer messages, consisting of a 4-byte float value. These are handled every update and use Conduit's built-in pooling support for inter-process transfer.
  \item Cell-cell communication messages, consisting of arbitrarily many 20-byte packets dispatched by genetic program execution. These are handled every 16 updates and use Conduit's built-in aggregation support for inter-process transfer.
  \item Environmental state messages, consisting of a 216-byte struct of data. These are handled every 8 updates and use conduit's built-in pooling support for inter-process transfer.
  \item Multicellular kin-group size detection messages, consisting of a 16-byte bitstring. These are handled every update and use Conduit's built-in pooling support for inter-process transfer.
\end{itemize}

Implementing all cell-cell interaction via Conduit-based messaging channels allows the simulation to be parallelized down to the granularity, potentially, of individual cells.
These messaging channels allow cells to communicate using the same interface whether they are placed within the same thread, across different threads, or across diffferent processes.
However, in practice, for this benchmarking we assign 3600 cells to each thread or process.
Because all cell-cell interactions occur via Conduit-based messaging channels, logically-neighboring cells can interact fully whether or not they are located on the same thread or process (albeit with potential irregularities due to best-effort limitations).
An alternate approach to evolving large populations might be an island model, where Conduit-based messaging channels would be used solely to exchange genomes between otherwise independent populations \citep{bennett1999building}.
However, we chose to instead parallelize DISHTINY as a unified spatial realm in order to enable parent-offspring interaction and leave the door open for future work with multicells that exceed the scope of an individual thread or process.


\subsection{ Graph Coloring Benchmark } \label{sec:graph_coloring_benchmark}

The graph coloring benchmark employs a graph coloring algorithm designed for distributed WLAN channel selection \citep{leith2012wlan}.
In this algorithm, nodes begin by randomly choosing a color.
Each computational update, nodes test for any neighbor with the same color.
If and only if a conflicting neighbor is detected, nodes randomly select another color.
The probability of selecting each possible color is stored in array associated with each node.
Before selecting a new color, the stored probability of selecting the current (conflicting) color is decreased by a multiplicative factor $b$.
We used $b=0.1$, as suggested by Leith et al.
Likewise, the stored probability of selecting all others is increased by a multiplicative factor.
Regardless of whether their color changed, nodes transmit their current color to their neighbor.

Our benchmarks focus on weak scalability, using a fixed problem size of 2048 graph nodes per thread or process.
These nodes were arranged in a two-dimensional grid topology where each node had three possible colors and four neighbors.
We implement the algorithm with a single Conduit communication layer carrying graph color as an unsigned integer.
We used Conduit's built-in pooling feature to consolidate color information into a single MPI message between pairs of communicating processes each update.
We performed five replicates, each with a five second simulation runtime.
Solution error was measured as the number of graph color conflicts remaining at the end of the benchmark.


\subsection{Asynchronicity Modes} \label{sec:asynchronicity_modes}

\subimport{submodule/2021-gecco-conduit/}{fig/asynchronicity_modes}

For both benchmarks, we compared performance across a spectrum of synchronization setting, which we term ``asynchronicity modes'' (Table \ref{tab:asynchronicity_modes}).
Asynchronicity mode 0 represents traditional fully-synchronous methodology.
Under this treatment, full barrier synchronization was performed between each computational update.
Asynchronicity mode 3 represents fully asynchronous methodology.
Under this treatment, individual threads or processes performed computational updates freely, incorporating input from other threads or processes on a fully best-effort basis.

During early development of the library, we discovered that unprocessed messages building up faster than they could be processed --- even if they were being skipped over to only get the latest message --- could degrade quality of service or even cause runtime instability.
We opted for MPI communication primitives that could consume many backlogged messages per call and increased buffer size to address these issues, but remained interested in the possibility of partial synchronization to clear potential message backlogs.
So, we included two partially-synchronized treatments: asynchronicity modes 1 and 2.

In asynchronicity mode 1, threads and processes alternated between performing computational updates for a fixed-time duration and executing a global barrier synchronization.
For the graph coloring benchmark, work was performed in 10ms chunks.
For the digital evolution benchmark, which is more computationally intensive, work was performed in 100ms chunks.
In asynchronicity mode 2, threads and processes executed global barrier synchronizations at predetermined time points.
In both experiments, global barrier synchronization occurred on each time a second of epoch time elapsed.

Finally, asynchronicity mode 4 disables all inter-thread and inter-process communication, including barrier synchronization.
We included this mode to isolate the impact on performance of communication between threads and processes from other factors potentially affecting performance, such as cache crowding.
In this run mode for the graph coloring benchmark, all calls to put messages into or pull messages out of ducts between processes or threads were skipped (except after the benchmark concluded, when assessing solution quality).
Because of its larger footprint, incorporating logic into the digital evolution simulation to disable all inter-thread and inter-process messaging was impractical.
Instead, we launched multiple instances of the simulation as fully-independent processes and measured performance of each.


\subsection{Quality of Service Metrics} \label{sec:quality-of-service-metrics}

\input{fig/quality-of-service-metric-definitions.tex}

The best-effort communication model eschews effort to insulate of computation from real-time message delivery dynamics.
Because these dynamics are difficult to predict \textit{a priori} and can bias computation, thorough, empirical runtime measurements are necessary to understand results of such computation.
To this end, we developed a suite of quality of service metrics.
Figure \ref{fig:quality-of-service-metric-definitions} provides space-time diagrams illustrating the metrics presented in this section.

For the purposes of these metrics, we assume that simulations proceed in an iterative fashion with alternating compute and communication phases.
For short, we refer to a single compute-communication cycle as a ``simstep.''
We derive formulas for metrics in terms of independent observations preceding and succeeding a ``snapshot'' window, during which the simulation and any associated best-effort communication proceeds unimpeded.
Snapshot observations are taken at one minute intervals over the course of each of our a replicate experiments.
The following section, \ref{sec:quality-of-service-experiments}, details the experimental apparatus used to generate quality of service metrics reported in this work.

\subsubsection{Simstep Period} \label{sec:simstep-period-metric}

We calculate the amount of wall-time elapsed per simulation update cycle (``Simstep Period'') during a snapshot window as
\begin{align*}
\frac{
  \mathrm{update\,count\,after} - \mathrm{update\,count\,before}
}{
  \mathrm{walltime\,after} - \mathrm{walltime\,before}
}.
\end{align*}
Figure \ref{fig:quality-of-service-metric-definitions-simstep-period} compares a scenario with low simstep period to a scenario with a higher simstep period.

\subsubsection{Simstep Latency} \label{sec:wall-time-latency-metric}

This metric reports the number of simulation iterations that elapse between message dispatch and message delivery.
Figure \ref{fig:quality-of-service-metric-definitions-latency} compares a scenario with low latency to a scenario with a higher latency.

To insulate against imperfect clock synchronization between processes, we estimate one-way wall-time latency from a round-trip measure.
As part of our instrumentation, each simulation element maintains an independent zero-initialized ``touch counter'' associated with every neighbor simulation element it communicates with.
Dispatched messages originating from each simulation element are bundled with the value of the unique touch counter associated with the target element's counter.
When a message is received back to the originating element from the target element, the touch counter is set to $1 + \mathrm{bundled\,touch\,count}$.
In this manner, the touch counter increments by two for each successful round trip completed.
(Because simulation elements are arranged as a toroidal mesh, all interaction between simulation elements is reciprocal.)

We therefore calculate one-way latency during a snapshot window as,
\begin{align*}
  \frac{
    \mathrm{update\,count\,after} - \mathrm{update\,count\,before}
  }{
    \min\Big( \mathrm{ touch\,count\,after } - \mathrm{ touch\,count\,before }, 1 \Big)
  }.
\end{align*}
Note that if no touches elapsed during the snapshot window, we make a best-case assumption that one might elapse immediately after the end of the snapshot window (i.e., we count at least one elapsed touch).

%TODO reference @rodsan's derivation and proof

\subsubsection{Wall-time Latency} \label{sec:simulation-time-latency-metric}

Wall-time latency is closely related to simstep latency, except that interpret time in terms of elapsed simulation updates instead of wall time.
To calculate wall-time latency we apply a conversion to simstep latency based on simstep period,
\begin{align*}
  \mathrm{simstep\,latency} \times \mathrm{simstep\,period}.
\end{align*}

This metric directly tells the real-time performance of message transmission.
Although it directly follows from the interaction between simstep period and wall-time latency, it complements simstep latency's convenient interpretation in terms of potential simulation mechanics (e.g., simulation elements tending to see data from two updates ago versus from ten).

In addition to simstep latency, Figure \ref{fig:quality-of-service-metric-definitions-latency} is also representative of wall-time latency --- the difference being interpretation of $y$ axis in terms of wall-time instead of elapsed simulation updates.

\subsubsection{Delivery Failure Rate} \label{sec:delivery-failure-rate-metric}

Delivery failure rate measures the fraction of messages sent that are dropped.
The only condition where messages are dropped is when a send buffer fills.
(Under the existing MPI-based implementation, messages that queue on the send buffer are guaranteed for delivery.)
So, we can calculate
\begin{align*}
  \frac{
    \mathrm{successful\,send\,count\,after} - \mathrm{successful\,send\,count\,before}
  }{
    \mathrm{attempted\,send\,count\,after} - \mathrm{attempted\,send\,count\,before}
  }.
\end{align*}

\subsubsection{Delivery Clumpiness} \label{sec:delivery-clumpiness-metric}

Delivery clumpiness seeks to quantify the extent to which message arrival is consolidated to a subset of message pull attempts.
That is, the extent to which independently dispatched messages arrive in bundles rather than as an even stream.

If messages all arrive in independent pull attempts, then clumpiness will be zero.
At the point where the pigeonhole principle applies ($\mathrm{num\,arriving\,messages} >= \mathrm{num\,pull\,attempts}$), clumpiness will also be zero so long as every pull attempt is laden.
If all messages arrive during a single pull attempt, then clumpiness will approach 1.

We formulate clumpiness as the compliment of steadiness.
(Reporting clumpiness provides a lower-is-better interpretation consistent with the rest of the quality of service metrics.)
Steadiness, in turn, stems from three component statistics,
\begin{align*}
\mathrm{num\,laden\,pulls\,elapsed} =& \mathrm{laden\,pull\,count\,after} \\
  &- \mathrm{laden\,pull\,count\,before} \\
\mathrm{num\,messages\,received} =& \mathrm{message\,count\,after} \\
  &- \mathrm{message\,count\,before} \\
\mathrm{num\,pulls\,attempted} =& \mathrm{pull\,attempt\,count\,after} \\
  &- \mathrm{pull\,attempt\,count\,before}
.
\end{align*}

Here, we refer to pull attempts that successfully retrieve a message as ``laden.''

We combine $\mathrm{num\,messages\,received}$ and $\mathrm{num\,pulls\,attempted}$ to derive,
\begin{align*}
  \mathrm{num\,opportunities\,for\,laden\,pulls} = \\
   \min\Big(\mathrm{num\,messages\,received}, \mathrm{num\,pulls\,attempted}\Big).
\end{align*}

Then, to calculate steadiness,
\begin{align*}
  \frac{
    \mathrm{num\,laden\,pulls\,elapsed}
  }{
    \mathrm{num\,opportunities\,for\,laden\,pulls}
  }.
\end{align*}

Finally, we find delivery clumpiness as $1 - \mathrm{steadiness}$.
Figure \ref{fig:quality-of-service-metric-definitions-clumpiness} compares a scenario with low clumpiness to a scenario with higher clumpiness.


\subsection{Quality of Service Experiments} \label{sec:quality-of-service-experiments}

Quality of service experiments executed the graph coloring algorithm described in Section \ref{sec:graph-coloring-benchmark}.
In order to maximize communication intensity, only one graph vertex was assigned per cpu.

Quality of service experiments were carried out on Michigan State University's High Performance Computing Center lac cluster, consisting of 28-core Intel(R) Xeon(R) CPU E5-2680 v4 \@ 2.40GHz nodes.
Ten experimental replicates were performed for each condition surveyed.
Slightly over five minutes of runtime was afforded to each replicate.
Over five minutes of runtime, snapshots were taken at one minute intervals.
The first snapshot was taken one minute after the beginning of runtime.

Snapshots lasted one second, with the graph coloring algorithm running fully unhampered during the entire snapshot.
This was accomplished by collecting and recording data via a separate thread.
That thread collected and recorded a first tranche of snapshot data, spin waited for one second, and then recorded a second tranche.
Because of underlying system being observed shifts during data collection (somewhat akin to photographic motion blur), some intuitive invariants --- like strictly non-negative delivery failure rates --- do not hold in some cases.
However, the magnitude of such violations is generally minor.
Further, because data collection procedures were consistent across treatments, statistical comparisons between treatments remain sound, even if direct interpretation of reported metrics should be taken with a grain of salt.

Snapshots were performed independently for each process at each timepoint.
So, for example, for two processes over the five minute window of a single replicate ten snapshots were collected.
For statistical tests comparing treatments, snapshots were aggregated by replicate by either mean or median, as appropriate.
For each quality of service statistic we estimate mean --- which captures effects of extreme-magnitude outliers --- and median --- which more closely represents typicalness --- across these window samples.
Statistical comparisons across treatment conditions are performed via regression.
We use ordinary least squares regression to analyze means \cite{geladi1986partial} and quantile regression to analyze medians \cite{koenker2001quantile}.
For comparisons between dichotomous, categorical treatment conditions, one condition is coded as 0 and the other as 1.
In the case of ordinary least squares regression, this boils down to an independent $t$-test.
Although quantile regression on categorical predictors is not precisely equivalent to a direct test on medians between two groups (i.e., Mood's median test), there is precedent for this approach \cite{petscher2014quantile, konstantopoulos2019using}.

Most statistics reported here can be calculated just as well in terms of incoming or outgoing messages.
That is, most statistics can be generated via data from instrumentation attached to message ``inlets'' or data from instrumentation attached to message ``outlets'' with no obvious reason to prefer one over the other.
Although ``inlet-'' and ``outlet-''derived statistics are nearly identical in all cases, for completeness we include both.


\subsection{Code, Data, and Reproducibility}

\subsubsection{Benchmarking Experiments} \label{sec:methods-code-data-reproducibility-benchmarking-experiments}

Benchmarking experiments were performed on Michigan State University's High Performance Computing Center, a cluster of hundreds of heterogeneous x86 nodes linked with InfiniBand interconnects.
For multithread experiments, benchmarks for each thread count were collected from the same node.
For multiprocess experiments, each processes was assigned to a distinct node in order to ensure results were representative of performance in a distributed context.
All multiprocess benchmarks were recorded from the same collection of nodes.
Hostnames are recorded for each benchmark data point.
For an exact accounting of hardware architectures used, these hostnames can be crossreferenced with a table included with the data that summarizes the cluster's node configurations.

Code for the distributed graph coloring benchmark is available at \url{https://github.com/mmore500/conduit} under \\ \texttt{demos/channel\_selection}.
Code for the digital evolution simulation benchmark is available at \url{https://github.com/mmore500/dishtiny}.
Exact versions of software used are recorded with each benchmark data point.
Data is available via the Open Science Framework at \url{https://osf.io/7jkgp/} \citep{foster2017open}.
A live, in-browser notebook for all reported statistics and data visualizations and is available via Binder at \url{https://mybinder.org/v2/gh/mmore500/conduit/binder?filepath=binder%2Fdate%3D2021%2Bproject%3D7jkgp} \citep{jupyter2018binder}.

\subsubsection{Quality of Service Experiments}

Quality of service experiments were performed on Quality of service experiments were carried out on Michigan State University's High Performance Computing Center lac cluster, consisting of 28-core Intel(R) Xeon(R) CPU E5-2680 v4 \@ 2.40GHz nodes.
All statistical comparisons are performed between observations from the same job allocation.
(Except in the case where intranode and internode configurations were compared, where experiments were performed on separate allocations using comparable nodes on the same cluster.)

Benchmarking experiments described in Section \ref{sec:methods-code-data-reproducibility-benchmarking-experiments} used a send/receive buffer size of 2.
However, due to the high communication intensity of the graph coloring problem with just one simulation element per CPU, quality of service experiments required a larger buffer size of 64 to maintain runtime stability.
In early work developing the Conduit library, we discovered that real-time messaging channels can enter a destabilizing positive feedback spiral when incoming messages take longer to handle (e.g., skip past or read) than sending messages.
Under such conditions, when a process exchanging messages from a partner process experiences a delay it sends fewer messages to that partner process.
Due to fewer incoming messages, the partner the partner process can update more rapidly, increasing incoming message load on the delayed process.
This effect can snowball the partnership intended for even, two-way message exchange into effectively a unilateral producer-consumer relationship where (potentially unbounded) work piles up on the consumer.
To interrupt such a scenario, we use the bulk message pull call \verb|MPI_Testsome| to ensure fast message consumption under backlogged conditions.
So, receiver workload remains closer to constant under high traffic situations (instead of having to pull messages down one-by-one).
Larger receive buffer size, as configured for the quality of service experiments, increases the effectiveness of the bulk message consumption countermeasure.

Code for the distributed graph coloring benchmark is available at \url{https://github.com/mmore500/conduit} under \\ \texttt{demos/channel\_selection}.
Exact versions of software used are recorded with each benchmark data point.
Data is available via the Open Science Framework at \url{https://osf.io/72k5n/} \citep{foster2017open}.
A live, in-browser notebook for all reported statistics and data visualizations is available via Binder at \url{https://mybinder.org/v2/gh/mmore500/conduit/binder?filepath=binder%2Fdate%3D2021%2Bproject%3D72k5n} \citep{jupyter2018binder}.



\section{Results and Discussion}

Sections \ref{sec:multithread-benchmarks} and \ref{sec:multiprocess-benchmarks} compare execution performance under the best-effort communication versus the perfect communication models.
In particular, both sections investigate how the impact of best-effort communication on performance relates to CPU count scale.
Section \ref{sec:multithread-benchmarks} covers multithreading and Section \ref{sec:multiprocess-benchmarks} covers multiprocessing.

The next sections investigate how system configuration affects quality of service.
Specifically, these sections cover the impact of
\begin{itemize}
  \item increasing compute workload per simulation update step (Section \ref{sec:computation-vs-communication}),
  \item within-node versus between-node process placement (Section \ref{sec:intranode-vs-internode}), and
  \item multithreading versus multiprocessing (Section \ref{sec:multithreading-vs-multiprocessing}).
\end{itemize}

Section \ref{sec:with-lac-417-vs-sans-lac-417} tests how inclusion of an apparently faulty node (i.e., that provided exceptionally poor quality of service) affects global quality of service.

Section \ref{sec:weak-scaling} tests how quality of service changes with CPU count.
This analysis fleshes out the performance-centric picture of best-effort scalability established in Sections \ref{sec:multithread-benchmarks} and \ref{sec:multiprocess-benchmarks}.

\subsection{Performance: Multithread Benchmarks}

\input{subtree/2021-gecco-conduit/fig/multithread_benchmarks.tex}

Figure \ref{fig:multithread_graph_coloring_update_rate} presents per-cpu algorithm update rate for the graph coloring benchmark at 1, 4, 16, and 64 threads.
Update rate performance decreased with increasing multithreading across all asynchronicity modes.
This performance degradation was rather severe --- per-cpu update rate decreased by 61\% between 1 and 4 threads and by about another 75\% between 4 and 64 threads.
Surprisingly, this issue appears largely unrelated to inter-thread communication, as it was also observed in asynchronicity mode 4, where all interthread communication is disabled.
Perhaps per-cpu update rate degradation under threading was induced by strain on a limited system resource like memory cache or access to the system clock (which was used to control run timing).
This unexpectedly severe phenomenon merits further investigation to fully in future work with this benchmark.

Nevertheless, we were able to observe significantly better performance of best-effort asynchronicity modes 1, 2, and 3 at high thread counts.
At 64 threads, these run modes significantly outperformed the fully-synchronized mode 0 ($p < 0.05$, non-overlapping 95\% confidence intervals).
Likewise, as shown in Figure \ref{fig:multithread_graph_coloring_solution_quality}, best-effort asynchronicity modes were able to deliver significantly better graph coloring solutions within the allotted compute time than the fully-synchronized mode 0 ($p < 0.05$, non-overlapping 95\% confidence intervals).

Figure \ref{fig:multithread_digital_evolution_update_rate} shows per-cpu algorithm update rate for the digital evolution benchmark at 1, 4, 16, and 64 threads.
Similarly to the graph coloring benchmark, update rate performance decreased with increasing multithreading across all asynchronicity modes --- including mode 4, which eschews inter-thread communication.
Even without communication between threads, with 64 threads each thread performed updates at only 61\% the rate of a lone thread.
At 64 threads, best-effort asynchronicity modes 1, 2, and 3 exhibit about 43\% the update-rate performance of a lone thread.
Although best-effort inter-thread communication only exhibits half the update-rate performance of completely decoupled execution at 64 threads, this update-rate performance is roughly $2.1\times$ that of the fully-synchronous mode 0.
Indeed, best-effort modes significantly outperform the fully-synchronous mode on the digital evolution benchmark at both 16 and 64 threads ($p < 0.05$, non-overlapping 95\% confidence intervals).


\subsection{Performance: Multiprocess Benchmarks}

\input{subtree/2021-gecco-conduit/fig/multiprocess_benchmarks.tex}

Figure \ref{fig:multiprocess_graph_coloring_update_rate} shows per-cpu algorithm update rate for the graph coloring benchmark at 1, 4, 16, and 64 processes.
Unlike the multithreaded benchmark, multiprocess graph coloring exhibits consistent update-rate performance across process counts under asynchronicity mode 4, where inter-thread communication is disabled.
This matches the expectation that, indeed, with comparable hardware a single process should exhibit the same mean performance as any number of completely decoupled processes.
At 64 processes, best-effort asynchronicity mode 3 exhibits about 63\% the update-rate performance of single-process execution.
This represents about an $7.8\times$ speedup compared to fully-synchronous mode 0.
Indeed, best-effort mode 3 enables significantly better per-cpu update rates at 4, 16, and 64 processes ($p < 0.05$, non-overlapping 95\% confidence intervals).

Likewise, shown in Figure \ref{fig:multiprocess_graph_coloring_solution_quality}, best-effort asynchronicity mode 3 yields significantly better graph-coloring results within the allotted time at 4, 16, and 64 processes ($p < 0.05$, non-overlapping 95\% confidence intervals).
Interestingly, partial-synchronization modes 1 and 2 exhibited highly inconsistent solution quality performance at 16 and 64 process count benchmarks.
Fixed-timepoint barrier sync (mode 2) had particularly poor performance performance at 64 processes (note the log-scale axis).
We suspect this was caused by a race condition where workers would assign sync points to different fixed points different based on slightly different startup times (i.e., process 0 syncs at seconds 0, 1, 2... while process 1 syncs at seconds 1, 2, 3..).

Figure \ref{fig:multiprocess_digital_evolution_update_rate} presents per-cpu algorithm update rate for the digital evolution benchmark at 1, 4, 16, and 64 processes.
Relative performance fares well at high process counts under this relatively computation-heavy workload, with 64-process fully best-effort simulation exhibiting about 92\% the update rate of single-process simulation.
This represents a $2.1\times$ speedup compared to the fully-synchronous run mode 0.
Best-effort significantly mode 3 outperforms the per-cpu update rate of fully-synchronous mode 0 at process counts 16 and 64 ($p < 0.05$, non-overlapping 95\% confidence intervals).


\subsection{Quality of Service: Computation vs. Communication}

Having shown performance benefits of best-effort communication on the graph coloring and digital evolution benchmarks in Sections \ref{sec:multithread-benchmarks} and \ref{sec:multiprocess-benchmarks}, we next seek to more fully characterize the best-effort approach using a holistic suite of proposed quality of service metrics.
This section evaluates how a simulation's ratio of communication intensity to computational work affects these quality of service metrics.
The graph coloring benchmark serves as our experimental system.

For this experiment, arbitrary compute work (detached from the underlying algorithm) was added to the simulation update process.
We used a call to the \texttt{std::mt19937} random number engine as a unit of compute work.
In microbenchmarks, we found that one work unit consumed about 35ns of walltime and 21ns of compute time.
We performed 5 treatments, adding \numprint{0}, \numprint{64}, \numprint{4096}, \numprint{262144}, or \numprint{16777216} units of compute work to the update process.
For each treatment, measurements were made on a pair of processes split across different nodes within the same cluster.

\subsubsection{Simstep Period}

Unsurprisingly, we found a direct relationship between per-update computational workload and the walltime required per computational update.
Supplementary Figures \ref{fig:computation-vs-communication-distribution-simstep-period-inlet-ns} and \ref{fig:computation-vs-communication-distribution-simstep-period-outlet-ns} depict the distribution of walltime per computational update across snapshots.
Once added compute work supersedes the light compute work already associated with the graph coloring algorithm update step (at around 64 work units), simstep period appears to scale in direct proportion with compute work.
Indeed, we found a significant positive relationship between both mean and median simstep period and added compute work (Supplementary Figures \ref{fig:computation-vs-communication-regression-simstep-period-inlet-ns} and \ref{fig:computation-vs-communication-regression-simstep-period-outlet-ns}).
At 0 units of added compute work, mean and median simstep period was 14.7 \SI{14.7}{\micro\second} (both inlet/outlet).
At \numprint{16777216} units of added compute work, mean simstep period was 614ms inlet/608ms outlet and median simstep period was 506ms inlet/508ms outlet.
Supplementary Tables \ref{tab:computation-vs-communication-ordinary-least-squares-regression} and \ref{tab:computation-vs-communication-quantile-regression} detail numerical results of these regressions.

\subsubsection{Simstep Latency}

Unsurprisingly, again, we observed a negative relationship between the number of simulation steps elapsed during message transit and added computational work.
Put simply, longer update steps provide more time for messages to transit.
Supplementary Figures \ref{fig:computation-vs-communication-distribution-latency-simsteps-inlet} and \ref{fig:computation-vs-communication-distribution-latency-simsteps-outlet} show the distribution of simstep latency across compute workloads.
With no added compute work, messages take between 20 and 100 simulation steps to transit (mean: 48.2 updates inlet/47.9 updates outlet; median: 42.5 updates inlet/outlet).
At maximum compute work per update, messages arrive at a median 1.00 update latency.
Regression analysis confirms a significant negative relationship between both mean and median log simstep latency and log added compute work (Supplementary Figures \ref{fig:computation-vs-communication-regression-latency-simsteps-inlet} and \ref{fig:computation-vs-communication-regression-latency-simsteps-outlet}).
Supplementary Tables \ref{tab:computation-vs-communication-ordinary-least-squares-regression} and \ref{tab:computation-vs-communication-quantile-regression} detail numerical results of these regressions.

%TODO use log latency simsteps for estimated statistics --- regressions are broken for raw input

\subsubsection{Walltime Latency}

Effects of log compute work on our measure of walltime latency highlight an important caveat in the interpretation of this metric.
At \numprint{0}, \numprint{64}, and \numprint{4096} work units, walltime latency measures $\approx 1$ ms (means: \SI{710}{\micro\second}, \SI{794}{\micro\second}, \SI{906}{\micro\second} inlet / \SI{706}{\micro\second}, \SI{782}{\micro\second}, \SI{899}{\micro\second}; medians: \SI{623}{\micro\second}, \SI{639}{\micro\second}, \SI{742}{\micro\second} inlet / \SI{622}{\micro\second}, \SI{642}{\micro\second}, \SI{733}{\micro\second} outlet).
However, once simstep period grows to $\approx 10$ ms at \numprint{262144} work units and (an order of magnitude in excess of walltime latency observed at low compute loads), walltime latency increases with added compute work.
At \numprint{16777216} compute work units, 1.00s inlet / 1.01s outlet median walltime latency is observed.
Because our computational model assumes on-demand message delivery with a communication phase only occurring once per simulation update, message transmission speed is fundamentally limited by simulation update period.
If a message is dispatched while its recipient is busy doing computational work, the soonest it can be received will be when that recipient completes the computational phase of its update.
In order to measure transmission time fully independent of delays due to on-demand delivery, additional instrumentation would be necessary.
However, when this latency is greater than a few simsteps, this measure is reasonably representative of message transmission time.

Supplementary Figures \ref{fig:computation-vs-communication-distribution-latency-walltime-inlet-ns} and \ref{fig:computation-vs-communication-distribution-latency-walltime-outlet-ns} show the distribution of walltime latency across computational workloads.
Supplementary Figures \ref{fig:computation-vs-communication-regression-latency-walltime-inlet-ns} and \ref{fig:computation-vs-communication-regression-latency-walltime-outlet-ns} summarize regression between walltime latency and added compute work.
Supplementary Tables \ref{tab:computation-vs-communication-ordinary-least-squares-regression} and \ref{tab:computation-vs-communication-quantile-regression} detail numerical results of those regressions.

\subsubsection{Delivery Clumpiness}

We observed a negative relationship between computation workload and delivery clumpiness.
At low computational intensity, we observed clumpiness greater than 0.95, meaning that fewer than 5\% of pull requests were laden with fresh messages (at 0 compute work mean: 0.96, median 0.96).
However, at high computational intensity clumpiness reached 0, indicating that messages arrived as a steady stream (at \numprint{16777216} compute work mean: 0.00, median 0.00).
Ostensibly, the reduction in clumpiness is due to increased real-time separation between dispatched messages.
Supplementary Figure \ref{fig:computation-vs-communication-distribution-delivery-clumpiness} shows the effect of computational workload on the distribution of observed clumpinesses.
We found a significant negative relationship between both mean and median clumpiness and computational intensity.
Supplementary Figure \ref{fig:computation-vs-communication-regression-delivery-clumpiness} visualizes these regressions and Supplementary Tables \ref{tab:computation-vs-communication-ordinary-least-squares-regression} and \ref{tab:computation-vs-communication-quantile-regression} provide numerical details.

\subsubsection{Delivery Failure Rate}

We did not observe any delivery failures across all replicates and all compute workloads.
So, compute workload had no observable effect on delivery reliability.
Interestingly, as discussed in Section \ref{sec:intranode-vs-internode}, we did observe some delivery failure under intranode conditions.
However, these experiments were conducted under internode conditions.
Supplementary Figure \ref{fig:computation-vs-communication-distribution-delivery-failure-rate} shows the distribution of delivery failure rates across computation workloads and Supplementary Figure \ref{fig:computation-vs-communication-regression-delivery-failure-rate} shows regressions of delivery failure rate of against computational workload
See Supplementary Tables \ref{tab:computation-vs-communication-ordinary-least-squares-regression} and \ref{tab:computation-vs-communication-quantile-regression} for numerical details.


\subsection{Quality of Service: Intranode vs. Internode}
\label{sec:intranode-vs-internode}

This section tests the effect of process assignment on best-effort quality of service, comparing multi-node and single-node assignments.
The graph coloring benchmark again serves as our experimental substrate.

For this experiment, processes were either assigned to the same node or were assigned to different nodes.
In both cases, we used two processes.

\subsubsection{Simstep Period}

Simstep period was significantly slower under internode conditions than under intranode conditions.

% When processes shared the same node, simstep period was around \SI{9}{\micro\second} (mean: \SI{9.07}{\micro\second} inlet / \SI{9.04}{\micro\second} outlet; median: \SI{9.11}{\micro\second} inlet / \SI{9.05}{\micro\second}).
When processes shared the same node, simstep period was around \SI{9}{\micro\second} (mean: \SI{9.06}{\micro\second}; median: \SI{9.08}{\micro\second}).
% Under internode conditions, simstep period was around \SI{14}{\micro\second} (mean: \SI{14.5}{\micro\second} inlet/outlet; median: \SI{14.5}{\micro\second} inlet / \SI{14.4}{\micro\second} outlet).
Under internode conditions, simstep period was around \SI{14}{\micro\second} (mean: \SI{14.5}{\micro\second}; median: \SI{14.4}{\micro\second}).
Supplementary Figures \ref{fig:intranode-vs-internode-distribution-simstep-period-inlet-ns} and \ref{fig:intranode-vs-internode-distribution-simstep-period-outlet-ns} depict the distribution of walltime per computational update across intranode and internode conditions.

This result presumably attributes to an increased walltime cost for calls to the MPI implementation backing internode communication compared to the MPI implementation backing intranode communication.
Although this effect is clearly detectable, its magnitude is modest given the minimal computational intensity of the simulation update step --- only $\approx 56\%$ more expensive than intranode dispatch.

Both mean and median simstep period increased significantly under internode conditions.
(Supplementary Figures \ref{fig:intranode-vs-internode-regression-simstep-period-inlet-ns} and \ref{fig:intranode-vs-internode-regression-simstep-period-outlet-ns} visualize these regressions and Supplementary Tables \ref{tab:intranode-vs-internode-ordinary-least-squares-regression} and \ref{tab:intranode-vs-internode-quantile-regression} detail numerical results.)

\subsubsection{Simstep Latency}

Significantly more simulation updates transpired during message transmission under internode condtions compared to intranode conditions.

Supplementary Figures \ref{fig:intranode-vs-internode-distribution-latency-simsteps-inlet} and \ref{fig:intranode-vs-internode-distribution-latency-simsteps-outlet} compares the distributions of simstep latency across these conditions.
% Simstep latency was around 1 update for intranode communication (mean: 1.01 updates inlet / 0.99 updates outlet; median 0.75 updates inlet / outlet) and around 40 updates for internode communication (mean: 41.8 updates inlet / 41.4 updates outlet; median: 37.4 updates inlet / outlet).
Simstep latency was around 1 update for intranode communication (mean: 1.00 updates; median 0.75 updates) and around 40 updates for internode communication (mean: 41.6 updates; median: 37.4 updates).

Regression analysis confirms the significant effect of process placement on simstep latency (Supplementary Figures \ref{fig:intranode-vs-internode-regression-latency-simsteps-inlet} and \ref{fig:intranode-vs-internode-regression-latency-simsteps-outlet}).
Supplementary Tables \ref{tab:intranode-vs-internode-ordinary-least-squares-regression} and \ref{tab:intranode-vs-internode-quantile-regression} detail numerical results of these regressions.

\subsubsection{Walltime Latency}

Significantly more walltime elapsed during message transmission under internode condtions compared to intranode conditions.

% Walltime latency was less than \SI{10}{\micro\second} for intranode communication (mean: \SI{7.72}{\micro\second} inlet / \SI{7.69}{\micro\second}; median: \SI{6.95}{\micro\second} inlet / \SI{6.93}{\micro\second}).
Walltime latency was less than \SI{10}{\micro\second} for intranode communication (mean: \SI{7.70}{\micro\second}; median: \SI{6.94}{\micro\second}).
% Internode communication had approximately $50\times$ greater walltime latency, at around \SI{500}{\micro\second} (mean: \SI{604}{\micro\second} inlet / \SI{596}{\micro\second} outlet; median: \SI{554}{\micro\second} inlet / \SI{548}{\micro\second} outlet).
Internode communication had approximately $50\times$ greater walltime latency, at around \SI{500}{\micro\second} (mean: \SI{600}{\micro\second}; median: \SI{551}{\micro\second}).

Supplementary Figures \ref{fig:intranode-vs-internode-distribution-latency-walltime-inlet-ns} and \ref{fig:intranode-vs-internode-distribution-latency-walltime-outlet-ns} show the distributions of walltime latency for intra- and inter-node communication.
Regression analysis confirmed a significant increase in walltime latency under inter-node communication (Supplementary Figures \ref{fig:intranode-vs-internode-regression-latency-walltime-inlet-ns}, \ref{fig:intranode-vs-internode-regression-latency-walltime-outlet-ns}; Supplementary Tables \ref{tab:intranode-vs-internode-ordinary-least-squares-regression} and \ref{tab:intranode-vs-internode-quantile-regression}).

\subsubsection{Delivery Clumpiness}

Delivery clumpiness was minimal under intranode communication and very high under internode communication.

Under intranode conditions, we observed a mean clumpiness value of 0.014 and a median of 0.002.
Under internode conditions, we observed mean and median clumpiness values of 0.96.
Supplementary Figures \ref{fig:intranode-vs-internode-distribution-delivery-clumpiness} and \ref{fig:intranode-vs-internode-distribution-delivery-clumpiness} show the distributions of clumpiness for intra- and inter-node communication.
Regression analysis confirmed a significant increase in clumpiness under inter-node communication (Supplementary Figures \ref{fig:intranode-vs-internode-regression-delivery-clumpiness}, \ref{fig:intranode-vs-internode-regression-delivery-clumpiness}; Supplementary Tables \ref{tab:intranode-vs-internode-ordinary-least-squares-regression} and \ref{tab:intranode-vs-internode-quantile-regression}).

\subsubsection{Delivery Failure Rate}

Somewhat counterintuitively, a significantly higher proportion of deliveries failed for intranode communication than for internode communication.

We observed a delivery failure rate of around 0.3 for intranode communication (mean: 0.33; median: 0.30) and no delivery failures for internode communication (mean: 0.00; median: 0.00).
In some intranode snapshot windows, we observed a delivery failure rate as high as 0.8.
Supplementary Figures \ref{fig:intranode-vs-internode-distribution-delivery-clumpiness} and \ref{fig:intranode-vs-internode-distribution-delivery-clumpiness} show the distributions of delivery failure rate for intra- and inter-node communication.

Because of Conduit's current MPI-based implementation, messages only drop when the underlying send buffer fills; queued messages are guaranteed for delivery.
Slower simstep period under internode allocation could improve stability of the send buffer due to more time, on average, between send attempts.
Underlying buffering or consolidation by the MPI backend for internode communication might also play a role by allowing data to be moved out of the userspace send buffer more promptly.

Regression analysis confirmed a significant increase in delivery failure under intra-node communication (Supplementary Figures \ref{fig:intranode-vs-internode-regression-delivery-clumpiness}, \ref{fig:intranode-vs-internode-regression-delivery-clumpiness}; Supplementary Tables \ref{tab:intranode-vs-internode-ordinary-least-squares-regression} and \ref{tab:intranode-vs-internode-quantile-regression}).


\subsection{Quality of Service: Multithreading vs. Multiprocessing}
\label{sec:multithreading-vs-multiprocessing}

This section compares best-effort quality of service under multithreading and multiprocessing schemes.
We hold hardware configuration constant by restricting multiprocessing to cores a single hardware node, as is the case for multithreading.
However, inter-process communication occurred via MPI calls while inter-thread communication occurring via shared memory access mediated by a C++ \texttt{std::mutex}.

The graph coloring benchmark again serves as our experimental system.
Both treatments used a single pair of CPUs.

\subsubsection{Simstep Period}

Multithreading enabled faster simulation update turnover than multiprocessing.

Under multithreading, simstep period was around \SI{5}{\micro\second} (mean: \SI{4.62}{\micro\second} inlet / \SI{4.60}{\micro\second} outlet; median: \SI{4.64}{\micro\second} inlet / outlet).
Simstep period for multiprocessing was around \SI{9}{\micro\second} (mean: \SI{9.01}{\micro\second} inlet / \SI{8.98}{\micro\second} outlet; median: \SI{9.07}{\micro\second} inlet / \SI{9.01}{\micro\second} outlet).
Supplementary Figures \ref{fig:multithreading-vs-multiprocessing-distribution-simstep-period-inlet-ns} and \ref{fig:multithreading-vs-multiprocessing-distribution-simstep-period-outlet-ns} depict the distribution of walltime per computational update for both multiprocessing and multithreading.
This result falls in line with expectations that interaction via shared memory incurs lower overhead than via MPI calls.

Regression analysis showed that both mean and median simstep period were significantly slower under multiprocessing compared to multithreading.
(Supplementary Figures \ref{fig:multithreading-vs-multiprocessing-regression-simstep-period-inlet-ns} and \ref{fig:multithreading-vs-multiprocessing-regression-simstep-period-outlet-ns} visualize these regressions and Supplementary Tables \ref{tab:multithreading-vs-multiprocessing-ordinary-least-squares-regression} and \ref{tab:multithreading-vs-multiprocessing-quantile-regression} detail numerical results.)

\subsubsection{Walltime Latency}

No significant difference in walltime latency was detected between multiprocessing and multithreading.

In the median case, walltime latency was approximately \SI{5}{\micro\second} for multithreading (\SI{5.05}{\micro\second} inlet / \SI{5.08}{\micro\second} outlet) and \SI{8}{\micro\second} for multiprocessing (\SI{7.84}{\micro\second} inlet / \SI{7.80}{\micro\second} outlet).
However, a pair of extreme outliers among snapshot windows --- with walltime latencies of approximately 12ms --- drove multithreading walltime latency much higher in the mean case (\SI{448}{\micro\second} inlet / \SI{454}{\micro\second} outlet).
In the mean case, multiprocessing walltime latency was \SI{8.65}{\micro\second} inlet / \SI{8.48}{\micro\second} outlet.

Cache invalidation or mutex contention provide possible explanations for the observed episodes of extreme multithreading latency, although magnitude on the order of milliseconds for such effects is surprising.
Multithreading appears to provide marginally lower latency service in the median case, but at the cost of vulnerability to extreme high-latency disruptions.

Supplementary Figures \ref{fig:multithreading-vs-multiprocessing-distribution-latency-walltime-inlet-ns} and \ref{fig:multithreading-vs-multiprocessing-distribution-latency-walltime-outlet-ns} show the distributions of walltime latency for multithread and multiprocess runs.
Regression analysis did not detect any significant difference in walltime latency between multithreading and multiprocessing (Supplementary Figures \ref{fig:multithreading-vs-multiprocessing-regression-latency-walltime-inlet-ns}, \ref{fig:multithreading-vs-multiprocessing-regression-latency-walltime-outlet-ns}; Supplementary Tables \ref{tab:multithreading-vs-multiprocessing-ordinary-least-squares-regression} and \ref{tab:multithreading-vs-multiprocessing-quantile-regression}).

\subsubsection{Simstep Latency}

No significant difference in simstep latency was detected between multiprocessing and multithreading.

In the median case, multiprocessing offered marginally lower simstep latency than multithreading.
Median simstep latency was 0.84 updates inlet/outlet under multiprocessing and 1.10 updates inlet / 1.11 updates outlet under multithreading.
However, just as for walltime latency, extreme magintude outliers ($\approx$ \numprint{2000} simsteps) boosted mean simstep latency for multithreading.
Mean simstep latency was 0.95 updates inlet / 0.94 updates outlet under multiprocessing and 77.0 updates inlet / 78.0 updates outlet under multithreading.
Supplementary Figures \ref{fig:multithreading-vs-multiprocessing-distribution-latency-simsteps-inlet} and \ref{fig:multithreading-vs-multiprocessing-distribution-latency-simsteps-outlet} compare the distributions of simstep latency across these conditions.

Direct measurements of simstep period and walltime latency suggest that faster simstep period, rather than slower walltime latency, explain the marginally higher simstep latency under multithreading.

Regression analysis detected no significant effect of threading versus processing on simstep latency in both the mean and median cases (Supplementary Figures \ref{fig:multithreading-vs-multiprocessing-regression-latency-simsteps-inlet} and \ref{fig:multithreading-vs-multiprocessing-regression-latency-simsteps-outlet}).
Supplementary Tables \ref{tab:multithreading-vs-multiprocessing-ordinary-least-squares-regression} and \ref{tab:multithreading-vs-multiprocessing-quantile-regression} detail numerical results of these regressions.

\subsubsection{Delivery Clumpiness}

Multithreading exhibited higher median clumpiness and greater variance in clumpiness than multiprocessing.

Under multithreading, clumpiness was nearly 1 within some snapshot windows and less than 0.1 within others.
Under multiprocessing, clumpiness was consistently less than 0.1.
Supplementary Figures \ref{fig:multithreading-vs-multiprocessing-distribution-delivery-clumpiness} and \ref{fig:multithreading-vs-multiprocessing-distribution-delivery-clumpiness} show the distributions of clumpiness under both multiprocessing and multithreading.
Multithreading median clumpiness was 0.54.
Multiprocessing median clumpiness was 0.03.
Multithreading and multiprocessing mean clumpinesses were 0.56 and 0.03, respectively.

Regression analysis confirmed a significantly greater clumpiness under both multithreading compared to multiprocessing (Supplementary Figures \ref{fig:multithreading-vs-multiprocessing-regression-delivery-clumpiness}, \ref{fig:multithreading-vs-multiprocessing-regression-delivery-clumpiness}; Supplementary Tables \ref{tab:multithreading-vs-multiprocessing-ordinary-least-squares-regression} and \ref{tab:multithreading-vs-multiprocessing-quantile-regression}).
This result falls in line with other results suggesting that multithreading providing less consistent quality of service than multiprocessing.

\subsubsection{Delivery Failure Rate}

We observed a higher proportion of deliveries fail for multiprocessing than for multithreading.
(This is as expected; the multithread implementation directly wrote updates to a piece of shared memory, so there was no send buffer to backlog and induce message drops.)

Multiprocessing exhibited both mean and median delivery failure rate of 0.38.
In individual multiprocessing snapshot windows, we observed a delivery failure rates ranging from less than 0.1 to as high as 0.7.
We observed no multithreaded delivery failures.
Supplementary Figures \ref{fig:multithreading-vs-multiprocessing-distribution-delivery-clumpiness} and \ref{fig:multithreading-vs-multiprocessing-distribution-delivery-clumpiness} show the distributions of delivery failure rate for multithreading and multiprocessing.

Regression analysis confirmed a significant increase in delivery failure under multiprocessing (Supplementary Figures \ref{fig:multithreading-vs-multiprocessing-regression-delivery-clumpiness}, \ref{fig:multithreading-vs-multiprocessing-regression-delivery-clumpiness}; Supplementary Tables \ref{tab:multithreading-vs-multiprocessing-ordinary-least-squares-regression} and \ref{tab:multithreading-vs-multiprocessing-quantile-regression}).


\subsection{Quality of Service: Weak Scaling} \label{sec:weak-scaling}

Sections \ref{sec:multiprocess-benchmarks} and \ref{sec:multithread-benchmarks} showed how best-effort communication could improve application performance, particularly when scaling up processor count.
Performance exhibited promising properties under multiprocess scaling, with overlapping performance estimate intervals for 16 and 64 processor counts on both surveyed benchmark problems.
This section aims to flesh out how increasing processor count affects a comprehensive suite of quality of service metrics.
Our particular interest is in which, if any, aspects of quality of service degrade under larger processing pools.

To address these questions, we performed weak scaling experiments on 16, 64, and 256 processes using the graph coloring benchmark.
To broaden the survey, we tested scaling under four treatments from the Cartesian product of two variables: processors allocated per node and simulation elements assigned per processor.
For the first variable, we tested scaling on allocations with each processor hosted on an independent node and allocations where each node hosted an average of four processors.
This allowed us to examine how quality of service fared in homogeneous network conditions, where all communication between processes was inter-node, compared to heteregeneous conditions, where some inter-process communication was inter-node and some was intra-node.
For the second variable, we tested with 2'048 simulation elements (``simels'') per processor (consistent with the benchmarking experiments performed in Sections \ref{sec:multiprocess-benchmarks} and \ref{sec:multithreading-benchmarks}) and just one simulation element per processor, maximizing communication relative to computation.

\subsubsection{Simstep Period}

Supplementary Figures \ref{fig:weak-scaling-distribution-simstep-period-inlet-ns} and \ref{fig:weak-scaling-distribution-simstep-period-outlet-ns} survey the distributions of simstep periods observed within snapshot windows.
Simstep period registers around \SI{80}{\micro\second} with one simel and around \SI{200}{\micro\second} with 2'048 simels.
However, on heterogeneous allocations (4 cpus per node) this metric is more variable, spanning up to an order of magnitude.
Outlier observations range up to around 10ms with 2'048 simels and up to slightly less than 100ms inlet / 4s outlet with 1 simel.

We performed an ordinary least squares (OLS) regression to test how mean simstep period changed with processor count.
In all cases except one simel per cpu with four cpus per node, mean simstep period increased significantly with processor count from 16 to 64 to 256.
However, from 64 to 256 processors mean simstep period only increased significantly with one simel per cpu and one cpu per node.
Between 64 and 256 processes, mean simstep period actually decreased significantly for runs with 2'048 simels per cpu.
Supplementary Figures \ref{fig:weak-scaling-regression-ols-simstep-period-inlet-ns} and \ref{fig:weak-scaling-regression-ols-simstep-period-outlet-ns} visualize reported OLS regressions.
Supplementary Tables \ref{tab:weak-scaling-simstep-period-inlet-ns-regression-ols} and \ref{tab:weak-scaling-simstep-period-outlet-ns-regression-ols} provide numerical details on reported OLS regressions.

Median simstep period exhibited the same relationships with processor count, tested with quartile regression.
Supplementary Figures \ref{fig:weak-scaling-regression-quantile-simstep-period-inlet-ns} and \ref{fig:weak-scaling-regression-quantile-simstep-period-outlet-ns} visualize corresponding quartile regressions.
Supplementary Tables \ref{tab:weak-scaling-simstep-period-inlet-ns-regression-quantile} and \ref{tab:weak-scaling-simstep-period-outlet-ns-regression-quantile} report numerical details on those quartile regressions.

\subsubsection{Walltime Latency}

Walltime latency sits at around \SI{500}{\micro\second} for one-simel runs and around 2ms for 2'048-simel runs.
However, variability is greater for heterogeneous (four cpus per node) allocations.
Extreme outliers of up to almost 100ms inlet/2s outlet occur in four cpus per node, one-simel runs.
In 256 process, 2'048-simel, one cpu per node runs, outliers of more than 10s occur.
Supplementary Figures \ref{fig:weak-scaling-distribution-latency-walltime-inlet-ns} and \ref{fig:weak-scaling-distribution-latency-walltime-outlet-ns} show the distribution of walltime latencies observed across run conditions.

We performed OLS regressions to test how mean walltime latency changed with processor count.
Over 16, 64, and 256 processes, mean walltime latency increased significantly with processor count only with 2'048 simels per cpu.
Between 64 and 256 processes, mean walltime latency increased significantly with processor count only for one cpu per node with 2'048 simels per cpu.
Supplementary Figures \ref{fig:weak-scaling-regression-ols-latency-walltime-inlet-ns} and \ref{fig:weak-scaling-regression-ols-latency-walltime-outlet-ns} show these regressions.
Supplementary Tables \ref{tab:weak-scaling-latency-walltime-inlet-ns-regression-ols} and \ref{tab:weak-scaling-latency-walltime-outlet-ns-regression-ols} provide numerical details.

Next, we performed quantile regressions to test how processor count affected median walltime latency.
Over 16, 64, and 256 processes, median walltime latency increased significantly only with 4 cpus per node and 2'048 simels per cpu.
Over 64 and 256 processes, there was no significant relationship between processor count and median walltime latency under any condition.
Supplementary Figures \ref{fig:weak-scaling-regression-quantile-latency-walltime-inlet-ns} and \ref{fig:weak-scaling-regression-quantile-latency-walltime-outlet-ns} show regressions performed.
Supplementary Tables \ref{tab:weak-scaling-latency-walltime-inlet-ns-regression-quantile} \ref{tab:weak-scaling-latency-walltime-outlet-ns-regression-quantile} provide numerical details.

\subsubsection{Simstep Latency}

Simstep latency sits around 7 updates for runs with one simel per cpu and around 1.2 updates for runs with 2'048 simels per cpu.
For runs with one simel per cpu, outlier snapshot windows reach up to 50 updates underhomogeneous allocations and up to almost 100 updates under heterogeneous allocations.
The 2'048 simels per cpu, one cpu per node, 256 process condition exhibited outliers of up to almost 8'000 update simstep latency.
Supplementary Figures \ref{fig:weak-scaling-distribution-latency-simsteps-inlet} and \ref{fig:weak-scaling-distribution-latency-simsteps-outlet}  show the distribution of simstep latencies observed across run conditions.

Over 16, 64, and 256 processes, mean simstep latency increased with process count only under 1 cpu per node, 2'048 simel per cpu conditions.
The same was true over just 64 to 256 processes.
Supplementary Figures \ref{fig:weak-scaling-regression-ols-latency-simsteps-inlet} and \ref{fig:weak-scaling-regression-ols-latency-simsteps-outlet} show the OLS regressions performed, with
Supplementary Tables \ref{tab:weak-scaling-latency-simsteps-inlet-regression-ols} and \ref{tab:weak-scaling-latency-simsteps-outlet-regression-ols} providing numerical details.

For median simstep latency, however, there was no condition where latency increased significantly with process count.
Supplementary Figures \ref{fig:weak-scaling-regression-quantile-latency-simsteps-inlet} and \ref{fig:weak-scaling-regression-quantile-latency-simsteps-outlet} show the quantile regressions performed, with Supplementary Tables \ref{tab:weak-scaling-latency-simsteps-inlet-regression-quantile} and \ref{tab:weak-scaling-latency-simsteps-outlet-regression-quantile} providing numerical details.

\subsubsection{Delivery Clumpiness}

For one-simel-per-cpu runs, median delivery clumpiness registered between 0.8 and 0.6.
On 2'048-simel-per-cpu runs, median delivery clumpiness was lower at around 0.4.
Supplementary Figure \ref{fig:weak-scaling-distribution-delivery-clumpiness}
shows the distribution of delivery clumpiness values observed across run conditions.

Using OLS regression, we found no evidence of mean clumpiness worsening with process count increases.
In fact, over 16, 64, and 256 processes clumpiness significantly decreased with process count in all conditions except four cpus per node with 2'048 simels per cpu.
Supplementary Figure \ref{fig:weak-scaling-regression-ols-delivery-clumpiness} and Supplementary Table \ref{tab:weak-scaling-delivery-clumpiness-regression-ols} detail regressions performed to test the relationship between mean clumpiness and process count.

Median delivery clumpiness exhibited the same relationships with processor count, tested with quartile regression.
Supplementary Figure \ref{fig:weak-scaling-regression-quantile-delivery-clumpiness} and Supplementary Table \ref{tab:weak-scaling-delivery-clumpiness-regression-quantile} detail regressions between median clumpiness and process count.

\subsubsection{Delivery Failure Rate}

Typical delivery failure rate was near-zero, except with one simel per cpu and four cpus per node where median delivery failure rate was approximately 0.1.
However, outlier delivery failure rates of up to 0.7 were observed with 1 cpu per node, 2'048 simels per cpu, and 256 processes.
Outlier delivery failure rates of up to 0.2 were observed with 4 cpus per node, 2'048 simels per cpu, and 256 processes.
Supplementary Figure \ref{fig:weak-scaling-distribution-delivery-failure-rate} shows the distribution of delivery failure rates observed across run conditions.

Mean delivery failure rate increased significantly between 64 and 256 processes with 1 cpu per node and 2'048 simels per cpu as well as with 4 cpus per node an 1 simel per cpu.
However, median delivery rate only increased significantly with processor count with 4 cpus per node and 1 simel per cpu.

Supplementary Figure \ref{fig:weak-scaling-regression-ols-delivery-failure-rate} and Supplementary Table \ref{tab:weak-scaling-delivery-failure-rate-regression-ols} detail the OLS regression testing mean delivery failure rate against processor count.
Supplementary Figure \ref{fig:weak-scaling-regression-quantile-delivery-failure-rate} and Supplementary Table \ref{tab:weak-scaling-delivery-failure-rate-regression-quantile} detail the quantile regression testing median delivery failure rate against processor count.


\subsection{Quality of Service: Faulty Hardware}
\label{sec:with-lac-417-vs-sans-lac-417}

The extreme magnitude of outliers for metrics reported in Section \ref{sec:weak-scaling} prompted further investigation of the conditions under which these outliers arose.
Closer inspection revealed that the most extreme outliers were all associated with snapshots on a single node: lac-417.

So, we acquired two separate 256 process allocations on the lac cluster: one including lac-417 and one excluding lac-417.

Supplementary Figures \ref{fig:with-lac-417-vs-sans-lac-417-distribution-simstep-period-inlet-ns}, \ref{fig:with-lac-417-vs-sans-lac-417-distribution-simstep-period-outlet-ns}, \ref{fig:with-lac-417-vs-sans-lac-417-distribution-latency-simsteps-inlet}, \ref{fig:with-lac-417-vs-sans-lac-417-distribution-latency-simsteps-outlet}, \ref{fig:with-lac-417-vs-sans-lac-417-distribution-latency-walltime-inlet-ns}, \ref{fig:with-lac-417-vs-sans-lac-417-distribution-latency-walltime-outlet-ns}, \ref{fig:with-lac-417-vs-sans-lac-417-distribution-delivery-clumpiness}, and \ref{fig:with-lac-417-vs-sans-lac-417-distribution-delivery-failure-rate} compare the distributions of quality of service metrics between allocations with and without lac-417.
Extreme outliers are present exclusively in the lac-417 allocation for walltime latency, updates latency, and delivery failure rate.
Otherwise, the metrics' distributions across snapshots are very similar between allocations.

Supplementary Figures \ref{fig:with-lac-417-vs-sans-lac-417-regression-simstep-period-inlet-ns}, \ref{fig:with-lac-417-vs-sans-lac-417-regression-simstep-period-outlet-ns}, \ref{fig:with-lac-417-vs-sans-lac-417-regression-latency-simsteps-inlet}, \ref{fig:with-lac-417-vs-sans-lac-417-regression-latency-simsteps-outlet}, \ref{fig:with-lac-417-vs-sans-lac-417-regression-latency-walltime-inlet-ns}, \ref{fig:with-lac-417-vs-sans-lac-417-regression-latency-walltime-outlet-ns}, \ref{fig:with-lac-417-vs-sans-lac-417-regression-delivery-clumpiness}, \ref{fig:with-lac-417-vs-sans-lac-417-regression-delivery-clumpiness}, and
\ref{fig:with-lac-417-vs-sans-lac-417-regression-delivery-failure-rate} chart OLS and quantile regressions of quality of service metrics on job composition.
Mean walltime latency, updates latency, and delivery failure rate are all significantly greater with lac-417.
Surprisingly, mean update period is significantly longer without lac-417.

However, there is no significant difference in median value for any quality of service metric between allocations including or excluding lac-417.
This stability of metric medians within allocations containing lac-417 --- which have significantly different means due to outlier values induced by the presence of lac-417 --- demonstrates how the best-effort system maintains overall quality of service stability despite defective or degraded components.

Supplementary Tables \ref{tab:with-lac-417-vs-sans-lac-417-ordinary-least-squares-regression} and \ref{tab:with-lac-417-vs-sans-lac-417-quantile-regression} provide numerical details on regressions reported above.



\section{Conclusion}

The fundamental motivation for best-effort communication is efficient scalability.
Our results confirm that best-effort communication can fulfill on this goal.

We found that the best-effort approach significantly increases performance at high CPU count.
This finding was consistent across the communication-intensive graph coloring benchmark and the computation-intensive digital evolution benchmark.
The computation-heavy digital evolution benchmark yielded the strongest scaling efficiency, achieving at 64 processes 92\% the update-rate of single-process execution.
We observed the greatest relative speedup under distributed communication-heavy workloads --- about $7.8\times$ on the graph coloring benchmark.
In the case of the graph coloring benchmark, we found that best-effort communication can help achieve tangibly better solution quality within a fixed time constraint.

Because real-time volatility affects the outcome of computation under the best-effort model, raw execution speed performance does not suffice to fully understand the consequences of the best-effort communication model.
In order to characterize the real-time dynamics under the best-effort model, we designed and measured a suite of quality of service metrics: simstep period, simstep latency, wall-time latency, delivery failure rate, and delivery clumpiness.

We performed several experiments to validate and characterize these metrics.
Comparing quality of service between multithreading and multiprocessing, we found that multithreading had lower runtime overhead cost but that multiprocessing reduced delivery erraticity, curbing especially extreme poor quality of service outlier events.
We found better quality of service, especially with respect to latency, for processes occupying the same node.
Finally, varying the ratio of computational work to communication, we found lower communication intensity associated with less volatile quality of service.

In order for best-effort communication to succeed in facilitating scale-up, median quality of service must stabilize with increasing CPU count.
Put another way, best-effort communication cannot succeed at scale if communication quality tends toward complete degradation.
In Section \ref{sec:weak-scaling}, we used weak scaling experiments to test the effect of scale-up on quality of service at 8, 64, and 256 processes.
Under a lower communication-intensivity task parameterization, we found that all median quality of service metrics were stable when scaling from 64 to 256 process.
Under maximal communication intensivity, we found in one case that median simstep period degraded from around \SI{80}{\micro\second} to around \SI{85}{\micro\second}.
In another case, median message delivery failure rate increased from around 7\% to around 9\%.
Such minor --- and, in most cases, nil --- degradation in median quality of service despite maximal communication intensivity bodes well for the viability of best-effort communication at scale.

Resilience is a second major motivating factor for best-effort computing.
In another promising result, we found that the presence of an apparently faulty compute node did not degrade median performance or quality of service.
Despite extreme quality of service degradation measured among that node and its clique, collective performance and quality of service remained steady.
In effect, the best-effort approach successfully decoupled global performance from the worst performer.
Such so-called ``straggler effects'' plague traditional approaches to large-scale high-performance computing \citep{aktacs2019straggler}, so avoiding them is a major boon.

Development of the Conduit library stemmed from a practical need for an abstract, prepackaged best-effort commmunication interface to support our digital evolution research.
Because real-time effects are fundamentally application-dependent and arise without any explicit in-program specification (and therefore may be unanticipated) it is important to be able to perform such quality of service profiling case-by-case in applications of best-effort communication.
The instrumentation used in these experiments is written as wrappers around the library's \texttt{Inlet} and \texttt{Outlet} classes that may be enabled via compile-time configuration switch.
This makes data generation for quality of service analysis trivial to perform in any system built with the Conduit library.
We hope that making this library and quality of service metrics available to the community can reduce domain expertise and programmability barriers to taking advantage of the best-effort communication model to efficiently leverage burgeoning parallel and distributed computing power.

In future work, it may be of interest to design systems that monitor and proactively react to real-time quality of service conditions.
For example, imposing a variable cost for cell-cell messaging to agents based on traffic levels or increasing per-update resource generation for agents on slow-running nodes.
We are eager to investigate how Conduit's best-effort communication model scales on much larger process counts on the order of thousands of cores.



\section*{Acknowledgment}

Thanks to Santiago Rodriguez Papa for contributing graphics illustrating quality of service metrics.
This research was supported in part by NSF grants DEB-1655715 and DBI-0939454 as well as by Michigan State University through the computational resources provided by the Institute for Cyber-Enabled Research.
This material is based upon work supported by the National Science Foundation Graduate Research Fellowship under Grant No. DGE-1424871.
Any opinions, findings, and conclusions or recommendations expressed in this material are those of the author(s) and do not necessarily reflect the views of the National Science Foundation.


\section*{References}


\clearpage

% \pragmaonce
% ^adapted from https://tex.stackexchange.com/a/195173

% adapted from https://tex.stackexchange.com/a/118450
\providecommand{\figweakscalingdistributionmacro}[2]{
\def\metric{#1}\def\metricslug{#2}\input{fig/weak-scaling/distribution/_figweakscalingdistributiontemplate.tex}
}


\figweakscalingdistributionmacro{Latency Walltime Inlet (ns)}{latency-walltime-inlet-ns}

% \pragmaonce
% ^adapted from https://tex.stackexchange.com/a/195173

% adapted from https://tex.stackexchange.com/a/118450
\providecommand{\figweakscalingdistributionmacro}[2]{
\def\metric{#1}\def\metricslug{#2}\input{fig/weak-scaling/distribution/_figweakscalingdistributiontemplate.tex}
}


\figweakscalingdistributionmacro{Latency Updates Outlet}{latency-simsteps-outlet}

% \pragmaonce
% ^adapted from https://tex.stackexchange.com/a/195173

% adapted from https://tex.stackexchange.com/a/118450
\providecommand{\figweakscalingdistributionmacro}[2]{
\def\metric{#1}\def\metricslug{#2}\input{fig/weak-scaling/distribution/_figweakscalingdistributiontemplate.tex}
}


\figweakscalingdistributionmacro{Latency Walltime Outlet (ns)}{latency-walltime-outlet-ns}

% \pragmaonce
% ^adapted from https://tex.stackexchange.com/a/195173

% adapted from https://tex.stackexchange.com/a/118450
\providecommand{\figweakscalingdistributionmacro}[2]{
\def\metric{#1}\def\metricslug{#2}\input{fig/weak-scaling/distribution/_figweakscalingdistributiontemplate.tex}
}


\figweakscalingdistributionmacro{Delivery Bunching}{delivery-clumpiness}

% \pragmaonce
% ^adapted from https://tex.stackexchange.com/a/195173

% adapted from https://tex.stackexchange.com/a/118450
\providecommand{\figweakscalingdistributionmacro}[2]{
\def\metric{#1}\def\metricslug{#2}\input{fig/weak-scaling/distribution/_figweakscalingdistributiontemplate.tex}
}


\figweakscalingdistributionmacro{Simstep Period Inlet (ns)}{simstep-period-inlet-ns}

% \pragmaonce
% ^adapted from https://tex.stackexchange.com/a/195173

% adapted from https://tex.stackexchange.com/a/118450
\providecommand{\figweakscalingdistributionmacro}[2]{
\def\metric{#1}\def\metricslug{#2}\input{fig/weak-scaling/distribution/_figweakscalingdistributiontemplate.tex}
}


\figweakscalingdistributionmacro{Latency Simsteps Inlet}{latency-simsteps-inlet}

% \pragmaonce
% ^adapted from https://tex.stackexchange.com/a/195173

% adapted from https://tex.stackexchange.com/a/118450
\providecommand{\figweakscalingdistributionmacro}[2]{
\def\metric{#1}\def\metricslug{#2}\input{fig/weak-scaling/distribution/_figweakscalingdistributiontemplate.tex}
}


\figweakscalingdistributionmacro{Simstep Period Outlet (ns)}{simstep-period-outlet-ns}

% \pragmaonce
% ^adapted from https://tex.stackexchange.com/a/195173

% adapted from https://tex.stackexchange.com/a/118450
\providecommand{\figweakscalingdistributionmacro}[2]{
\def\metric{#1}\def\metricslug{#2}\input{fig/weak-scaling/distribution/_figweakscalingdistributiontemplate.tex}
}


\figweakscalingdistributionmacro{Delivery Failure Rate}{delivery-failure-rate}


% \pragmaonce
% ^adapted from https://tex.stackexchange.com/a/195173

% adapted from https://tex.stackexchange.com/a/118450
\providecommand{\figweakscalingregressionolsmacro}[2]{
\def\metric{#1}\def\metricslug{#2}\input{fig/weak-scaling/regression-ols/_figweakscalingregressionolstemplate.tex}
}


\figweakscalingregressionolsmacro{Latency Walltime Inlet (ns)}{latency-walltime-inlet-ns}

% \pragmaonce
% ^adapted from https://tex.stackexchange.com/a/195173

% adapted from https://tex.stackexchange.com/a/118450
\providecommand{\figweakscalingregressionolsmacro}[2]{
\def\metric{#1}\def\metricslug{#2}\input{fig/weak-scaling/regression-ols/_figweakscalingregressionolstemplate.tex}
}


\figweakscalingregressionolsmacro{Latency Simsteps Outlet}{latency-simsteps-outlet}

% \pragmaonce
% ^adapted from https://tex.stackexchange.com/a/195173

% adapted from https://tex.stackexchange.com/a/118450
\providecommand{\figweakscalingregressionolsmacro}[2]{
\def\metric{#1}\def\metricslug{#2}\input{fig/weak-scaling/regression-ols/_figweakscalingregressionolstemplate.tex}
}


\figweakscalingregressionolsmacro{Latency Walltime Outlet (ns)}{latency-walltime-outlet-ns}

% \pragmaonce
% ^adapted from https://tex.stackexchange.com/a/195173

% adapted from https://tex.stackexchange.com/a/118450
\providecommand{\figweakscalingregressionolsmacro}[2]{
\def\metric{#1}\def\metricslug{#2}\input{fig/weak-scaling/regression-ols/_figweakscalingregressionolstemplate.tex}
}


\figweakscalingregressionolsmacro{Delivery Bunching}{delivery-clumpiness}

% \pragmaonce
% ^adapted from https://tex.stackexchange.com/a/195173

% adapted from https://tex.stackexchange.com/a/118450
\providecommand{\figweakscalingregressionolsmacro}[2]{
\def\metric{#1}\def\metricslug{#2}\input{fig/weak-scaling/regression-ols/_figweakscalingregressionolstemplate.tex}
}


\figweakscalingregressionolsmacro{Simstep Period Inlet (ns)}{simstep-period-inlet-ns}

% \pragmaonce
% ^adapted from https://tex.stackexchange.com/a/195173

% adapted from https://tex.stackexchange.com/a/118450
\providecommand{\figweakscalingregressionolsmacro}[2]{
\def\metric{#1}\def\metricslug{#2}\input{fig/weak-scaling/regression-ols/_figweakscalingregressionolstemplate.tex}
}


\figweakscalingregressionolsmacro{Latency Updates Inlet}{latency-simsteps-inlet}

% \pragmaonce
% ^adapted from https://tex.stackexchange.com/a/195173

% adapted from https://tex.stackexchange.com/a/118450
\providecommand{\figweakscalingregressionolsmacro}[2]{
\def\metric{#1}\def\metricslug{#2}\input{fig/weak-scaling/regression-ols/_figweakscalingregressionolstemplate.tex}
}


\figweakscalingregressionolsmacro{Simstep Period Outlet (ns)}{simstep-period-outlet-ns}

% \pragmaonce
% ^adapted from https://tex.stackexchange.com/a/195173

% adapted from https://tex.stackexchange.com/a/118450
\providecommand{\figweakscalingregressionolsmacro}[2]{
\def\metric{#1}\def\metricslug{#2}\input{fig/weak-scaling/regression-ols/_figweakscalingregressionolstemplate.tex}
}


\figweakscalingregressionolsmacro{Delivery Failure Rate}{delivery-failure-rate}


\pragmaonce
% ^adapted from https://tex.stackexchange.com/a/195173

% adapted from https://tex.stackexchange.com/a/118450
\newcommand{\figweakscalingregressionquantilemacro}[2]{
\def\metric{#1}\def\metricslug{#2}\input{fig/weak-scaling/regression-quantile/_figweakscalingregressionquantiletemplate.tex}
}


\figweakscalingregressionquantilemacro{Latency Walltime Inlet (ns)}{latency-walltime-inlet-ns}

\pragmaonce
% ^adapted from https://tex.stackexchange.com/a/195173

% adapted from https://tex.stackexchange.com/a/118450
\newcommand{\figweakscalingregressionquantilemacro}[2]{
\def\metric{#1}\def\metricslug{#2}\input{fig/weak-scaling/regression-quantile/_figweakscalingregressionquantiletemplate.tex}
}


\figweakscalingregressionquantilemacro{Latency Updates Outlet}{latency-simsteps-outlet}

\pragmaonce
% ^adapted from https://tex.stackexchange.com/a/195173

% adapted from https://tex.stackexchange.com/a/118450
\newcommand{\figweakscalingregressionquantilemacro}[2]{
\def\metric{#1}\def\metricslug{#2}\input{fig/weak-scaling/regression-quantile/_figweakscalingregressionquantiletemplate.tex}
}


\figweakscalingregressionquantilemacro{Latency Walltime Outlet (ns)}{latency-walltime-outlet-ns}

\pragmaonce
% ^adapted from https://tex.stackexchange.com/a/195173

% adapted from https://tex.stackexchange.com/a/118450
\newcommand{\figweakscalingregressionquantilemacro}[2]{
\def\metric{#1}\def\metricslug{#2}\input{fig/weak-scaling/regression-quantile/_figweakscalingregressionquantiletemplate.tex}
}


\figweakscalingregressionquantilemacro{Delivery Bunching}{delivery-clumpiness}

\pragmaonce
% ^adapted from https://tex.stackexchange.com/a/195173

% adapted from https://tex.stackexchange.com/a/118450
\newcommand{\figweakscalingregressionquantilemacro}[2]{
\def\metric{#1}\def\metricslug{#2}\input{fig/weak-scaling/regression-quantile/_figweakscalingregressionquantiletemplate.tex}
}


\figweakscalingregressionquantilemacro{Simstep Period Inlet (ns)}{simstep-period-inlet-ns}

\pragmaonce
% ^adapted from https://tex.stackexchange.com/a/195173

% adapted from https://tex.stackexchange.com/a/118450
\newcommand{\figweakscalingregressionquantilemacro}[2]{
\def\metric{#1}\def\metricslug{#2}\input{fig/weak-scaling/regression-quantile/_figweakscalingregressionquantiletemplate.tex}
}


\figweakscalingregressionquantilemacro{Latency Simsteps Inlet}{latency-simsteps-inlet}

\pragmaonce
% ^adapted from https://tex.stackexchange.com/a/195173

% adapted from https://tex.stackexchange.com/a/118450
\newcommand{\figweakscalingregressionquantilemacro}[2]{
\def\metric{#1}\def\metricslug{#2}\input{fig/weak-scaling/regression-quantile/_figweakscalingregressionquantiletemplate.tex}
}


\figweakscalingregressionquantilemacro{Update Period Outlet (ns)}{simstep-period-outlet-ns}

\pragmaonce
% ^adapted from https://tex.stackexchange.com/a/195173

% adapted from https://tex.stackexchange.com/a/118450
\newcommand{\figweakscalingregressionquantilemacro}[2]{
\def\metric{#1}\def\metricslug{#2}\input{fig/weak-scaling/regression-quantile/_figweakscalingregressionquantiletemplate.tex}
}


\figweakscalingregressionquantilemacro{Delivery Failure Rate}{delivery-failure-rate}


\pragmaonce
% ^adapted from https://tex.stackexchange.com/a/195173

% adapted from https://tex.stackexchange.com/a/118450
\providecommand{\tabweakscalingregressionmacro}[5]{
\def\regression{#1}\def\regressionslug{#2}\def\regressionlongslug{#3}\def\metric{#4}\def\metricslug{#5}\input{tab/weak-scaling/_tabweakscalingregressiontemplate.tex}
}


\tabweakscalingregressionmacro{%
Ordinary Least Squares Regression%
}{%
regression-ols%
}{%
ordinary-least-squares-regression%
}{%
Latency Walltime Inlet (ns)%
}{%
latency-walltime-inlet-ns%
}

\pragmaonce
% ^adapted from https://tex.stackexchange.com/a/195173

% adapted from https://tex.stackexchange.com/a/118450
\providecommand{\tabweakscalingregressionmacro}[5]{
\def\regression{#1}\def\regressionslug{#2}\def\regressionlongslug{#3}\def\metric{#4}\def\metricslug{#5}\input{tab/weak-scaling/_tabweakscalingregressiontemplate.tex}
}


\tabweakscalingregressionmacro{Ordinary Least Squares Regression}{ordinary-least-squares-regression}{Latency Simsteps Outlet}{latency-simsteps-outlet}

\pragmaonce
% ^adapted from https://tex.stackexchange.com/a/195173

% adapted from https://tex.stackexchange.com/a/118450
\providecommand{\tabweakscalingregressionmacro}[5]{
\def\regression{#1}\def\regressionslug{#2}\def\regressionlongslug{#3}\def\metric{#4}\def\metricslug{#5}\input{tab/weak-scaling/_tabweakscalingregressiontemplate.tex}
}


\tabweakscalingregressionmacro{Ordinary Least Squares Regression}{regression-ols}{Latency Walltime Outlet (ns)}{latency-walltime-outlet-ns}

\pragmaonce
% ^adapted from https://tex.stackexchange.com/a/195173

% adapted from https://tex.stackexchange.com/a/118450
\providecommand{\tabweakscalingregressionmacro}[5]{
\def\regression{#1}\def\regressionslug{#2}\def\regressionlongslug{#3}\def\metric{#4}\def\metricslug{#5}\input{tab/weak-scaling/_tabweakscalingregressiontemplate.tex}
}


\tabweakscalingregressionmacro{%
Ordinary Least Squares Regression%
}{%
regression-ols%
}{%
ordinary-least-squares-regression%
}{%
Delivery Bunching%
}{%
delivery-clumpiness%
}

\pragmaonce
% ^adapted from https://tex.stackexchange.com/a/195173

% adapted from https://tex.stackexchange.com/a/118450
\providecommand{\tabweakscalingregressionmacro}[5]{
\def\regression{#1}\def\regressionslug{#2}\def\regressionlongslug{#3}\def\metric{#4}\def\metricslug{#5}\input{tab/weak-scaling/_tabweakscalingregressiontemplate.tex}
}


\tabweakscalingregressionmacro{%
Ordinary Least Squares Regression%
}{%
regression-ols%
}{%
ordinary-least-squares-regression%
}{%
Simstep Period Inlet (ns)%
}{%
simstep-period-inlet-ns%
}

\pragmaonce
% ^adapted from https://tex.stackexchange.com/a/195173

% adapted from https://tex.stackexchange.com/a/118450
\providecommand{\tabweakscalingregressionmacro}[5]{
\def\regression{#1}\def\regressionslug{#2}\def\regressionlongslug{#3}\def\metric{#4}\def\metricslug{#5}\input{tab/weak-scaling/_tabweakscalingregressiontemplate.tex}
}


\tabweakscalingregressionmacro{%
Ordinary Least Squares Regression%
}{%
regression-ols%
}{%
ordinary-least-squares-regression%
}{%
Latency Simsteps Inlet%
}{%
latency-simsteps-inlet%
}

\pragmaonce
% ^adapted from https://tex.stackexchange.com/a/195173

% adapted from https://tex.stackexchange.com/a/118450
\providecommand{\tabweakscalingregressionmacro}[5]{
\def\regression{#1}\def\regressionslug{#2}\def\regressionlongslug{#3}\def\metric{#4}\def\metricslug{#5}\input{tab/weak-scaling/_tabweakscalingregressiontemplate.tex}
}


\tabweakscalingregressionmacro{%
Ordinary Least Squares Regression%
}{%
regression-ols%
}{%
ordinary-least-squares-regression%
}{%
Simstep Period Outlet (ns)%
}{%
simstep-period-outlet-ns%
}

\pragmaonce
% ^adapted from https://tex.stackexchange.com/a/195173

% adapted from https://tex.stackexchange.com/a/118450
\providecommand{\tabweakscalingregressionmacro}[5]{
\def\regression{#1}\def\regressionslug{#2}\def\regressionlongslug{#3}\def\metric{#4}\def\metricslug{#5}\input{tab/weak-scaling/_tabweakscalingregressiontemplate.tex}
}


\tabweakscalingregressionmacro{Ordinary Least Squares Regression}{regression-ols}{Delivery Failure Rate}{delivery-failure-rate}



\pragmaonce
% ^adapted from https://tex.stackexchange.com/a/195173

% adapted from https://tex.stackexchange.com/a/118450
\providecommand{\tabweakscalingregressionmacro}[5]{
\def\regression{#1}\def\regressionslug{#2}\def\regressionlongslug{#3}\def\metric{#4}\def\metricslug{#5}\input{tab/weak-scaling/_tabweakscalingregressiontemplate.tex}
}


\tabweakscalingregressionmacro{Quantile Regression}{regression-quantile}{Latency Walltime Inlet (ns)}{latency-walltime-inlet-ns}

\pragmaonce
% ^adapted from https://tex.stackexchange.com/a/195173

% adapted from https://tex.stackexchange.com/a/118450
\providecommand{\tabweakscalingregressionmacro}[5]{
\def\regression{#1}\def\regressionslug{#2}\def\regressionlongslug{#3}\def\metric{#4}\def\metricslug{#5}\input{tab/weak-scaling/_tabweakscalingregressiontemplate.tex}
}


\tabweakscalingregressionmacro{Quantile Regression}{regression-quantile}{Latency Simsteps Outlet}{latency-simsteps-outlet}

\pragmaonce
% ^adapted from https://tex.stackexchange.com/a/195173

% adapted from https://tex.stackexchange.com/a/118450
\providecommand{\tabweakscalingregressionmacro}[5]{
\def\regression{#1}\def\regressionslug{#2}\def\regressionlongslug{#3}\def\metric{#4}\def\metricslug{#5}\input{tab/weak-scaling/_tabweakscalingregressiontemplate.tex}
}


\tabweakscalingregressionmacro{Quantile Regression}{regression-quantile}{Latency Walltime Outlet (ns)}{latency-walltime-outlet-ns}

\pragmaonce
% ^adapted from https://tex.stackexchange.com/a/195173

% adapted from https://tex.stackexchange.com/a/118450
\providecommand{\tabweakscalingregressionmacro}[5]{
\def\regression{#1}\def\regressionslug{#2}\def\regressionlongslug{#3}\def\metric{#4}\def\metricslug{#5}\input{tab/weak-scaling/_tabweakscalingregressiontemplate.tex}
}


\tabweakscalingregressionmacro{%
Quantile Regression%
}{%
regression-quantile%
}{%
quantile-regression%
}{%
Delivery Bunching%
}{%
delivery-clumpiness%
}

\pragmaonce
% ^adapted from https://tex.stackexchange.com/a/195173

% adapted from https://tex.stackexchange.com/a/118450
\providecommand{\tabweakscalingregressionmacro}[5]{
\def\regression{#1}\def\regressionslug{#2}\def\regressionlongslug{#3}\def\metric{#4}\def\metricslug{#5}\input{tab/weak-scaling/_tabweakscalingregressiontemplate.tex}
}


\tabweakscalingregressionmacro{%
Quantile Regression%
}{%
regression-quantile%
}{%
quantile-regression%
}{%
Simsteps Period Inlet (ns)%
}{%
simstep-period-inlet-ns%
}

\pragmaonce
% ^adapted from https://tex.stackexchange.com/a/195173

% adapted from https://tex.stackexchange.com/a/118450
\providecommand{\tabweakscalingregressionmacro}[5]{
\def\regression{#1}\def\regressionslug{#2}\def\regressionlongslug{#3}\def\metric{#4}\def\metricslug{#5}\input{tab/weak-scaling/_tabweakscalingregressiontemplate.tex}
}


\tabweakscalingregressionmacro{%
Quantile Regression%
}{%
regression-quantile%
}{%
quantile-regression%
}{%
Latency Updates Inlet%
}{%
latency-simsteps-inlet%
}

\pragmaonce
% ^adapted from https://tex.stackexchange.com/a/195173

% adapted from https://tex.stackexchange.com/a/118450
\providecommand{\tabweakscalingregressionmacro}[5]{
\def\regression{#1}\def\regressionslug{#2}\def\regressionlongslug{#3}\def\metric{#4}\def\metricslug{#5}\input{tab/weak-scaling/_tabweakscalingregressiontemplate.tex}
}


\tabweakscalingregressionmacro{%
Quantile Regression%
}{%
regression-quantile%
}{%
quantile-regression%
}{%
Simstep Period Outlet (ns)%
}{%
simstep-period-outlet-ns%
}

\pragmaonce
% ^adapted from https://tex.stackexchange.com/a/195173

% adapted from https://tex.stackexchange.com/a/118450
\providecommand{\tabweakscalingregressionmacro}[5]{
\def\regression{#1}\def\regressionslug{#2}\def\regressionlongslug{#3}\def\metric{#4}\def\metricslug{#5}\input{tab/weak-scaling/_tabweakscalingregressiontemplate.tex}
}


\tabweakscalingregressionmacro{%
Quantile Regression%
}{%
regression-quantile%
}{%
quantile-regression%
}{%
Delivery Failure Rate%
}{%
delivery-failure-rate%
}


\clearpage

%\pragmaonce
% ^adapted from https://tex.stackexchange.com/a/195173

% adapted from https://tex.stackexchange.com/a/118450
\providecommand{\figcomputationvscommunicationdistributionmacro}[2]{
\def\metric{#1}\def\metricslug{#2}\input{fig/computation-vs-communication/distribution/_figcomputationvscommunicationdistributiontemplate.tex}
}


\figcomputationvscommunicationdistributionmacro{Latency Walltime Inlet (ns)}{latency-walltime-inlet-ns}

%\pragmaonce
% ^adapted from https://tex.stackexchange.com/a/195173

% adapted from https://tex.stackexchange.com/a/118450
\providecommand{\figcomputationvscommunicationdistributionmacro}[2]{
\def\metric{#1}\def\metricslug{#2}\input{fig/computation-vs-communication/distribution/_figcomputationvscommunicationdistributiontemplate.tex}
}


\figcomputationvscommunicationdistributionmacro{Latency Simsteps Outlet}{latency-simsteps-outlet}

%\pragmaonce
% ^adapted from https://tex.stackexchange.com/a/195173

% adapted from https://tex.stackexchange.com/a/118450
\providecommand{\figcomputationvscommunicationdistributionmacro}[2]{
\def\metric{#1}\def\metricslug{#2}\input{fig/computation-vs-communication/distribution/_figcomputationvscommunicationdistributiontemplate.tex}
}


\figcomputationvscommunicationdistributionmacro{Latency Walltime Outlet (ns)}{latency-walltime-outlet-ns}

%\pragmaonce
% ^adapted from https://tex.stackexchange.com/a/195173

% adapted from https://tex.stackexchange.com/a/118450
\providecommand{\figcomputationvscommunicationdistributionmacro}[2]{
\def\metric{#1}\def\metricslug{#2}\input{fig/computation-vs-communication/distribution/_figcomputationvscommunicationdistributiontemplate.tex}
}


\figcomputationvscommunicationdistributionmacro{Delivery Bunching}{delivery-clumpiness}

%\pragmaonce
% ^adapted from https://tex.stackexchange.com/a/195173

% adapted from https://tex.stackexchange.com/a/118450
\providecommand{\figcomputationvscommunicationdistributionmacro}[2]{
\def\metric{#1}\def\metricslug{#2}\input{fig/computation-vs-communication/distribution/_figcomputationvscommunicationdistributiontemplate.tex}
}


\figcomputationvscommunicationdistributionmacro{Update Period Inlet (ns)}{simstep-period-inlet-ns}

%\pragmaonce
% ^adapted from https://tex.stackexchange.com/a/195173

% adapted from https://tex.stackexchange.com/a/118450
\providecommand{\figcomputationvscommunicationdistributionmacro}[2]{
\def\metric{#1}\def\metricslug{#2}\input{fig/computation-vs-communication/distribution/_figcomputationvscommunicationdistributiontemplate.tex}
}


\figcomputationvscommunicationdistributionmacro{Latency Updates Inlet}{latency-simsteps-inlet}

%\pragmaonce
% ^adapted from https://tex.stackexchange.com/a/195173

% adapted from https://tex.stackexchange.com/a/118450
\providecommand{\figcomputationvscommunicationdistributionmacro}[2]{
\def\metric{#1}\def\metricslug{#2}\input{fig/computation-vs-communication/distribution/_figcomputationvscommunicationdistributiontemplate.tex}
}


\figcomputationvscommunicationdistributionmacro{Update Period Outlet (ns)}{simstep-period-outlet-ns}

%\pragmaonce
% ^adapted from https://tex.stackexchange.com/a/195173

% adapted from https://tex.stackexchange.com/a/118450
\providecommand{\figcomputationvscommunicationdistributionmacro}[2]{
\def\metric{#1}\def\metricslug{#2}\input{fig/computation-vs-communication/distribution/_figcomputationvscommunicationdistributiontemplate.tex}
}


\figcomputationvscommunicationdistributionmacro{Delivery Failure Rate}{delivery-failure-rate}


%\pragmaonce
% ^adapted from https://tex.stackexchange.com/a/195173

% adapted from https://tex.stackexchange.com/a/118450
\providecommand{\figcomputationvscommunicationregressionmacro}[4]{
\def\metric{#1}\def\metricslug{#2}\def\xvarslug{#3}\def\yvarprefix{#4}\input{fig/computation-vs-communication/regression/_figcomputationvscommunicationregressiontemplate.tex}
}


\figcomputationvscommunicationregressionmacro{Latency Walltime Inlet (ns)}{latency-walltime-inlet-ns}{log-compute-work}{log-}

%\pragmaonce
% ^adapted from https://tex.stackexchange.com/a/195173

% adapted from https://tex.stackexchange.com/a/118450
\providecommand{\figcomputationvscommunicationregressionmacro}[4]{
\def\metric{#1}\def\metricslug{#2}\def\xvarslug{#3}\def\yvarprefix{#4}\input{fig/computation-vs-communication/regression/_figcomputationvscommunicationregressiontemplate.tex}
}


\figcomputationvscommunicationregressionmacro{Latency Simsteps Outlet}{latency-simsteps-outlet}{log-compute-work}{log-}

%\pragmaonce
% ^adapted from https://tex.stackexchange.com/a/195173

% adapted from https://tex.stackexchange.com/a/118450
\providecommand{\figcomputationvscommunicationregressionmacro}[4]{
\def\metric{#1}\def\metricslug{#2}\def\xvarslug{#3}\def\yvarprefix{#4}\input{fig/computation-vs-communication/regression/_figcomputationvscommunicationregressiontemplate.tex}
}


\figcomputationvscommunicationregressionmacro{Latency Walltime Outlet (ns)}{latency-walltime-outlet-ns}{log-compute-work}{log-}

%\pragmaonce
% ^adapted from https://tex.stackexchange.com/a/195173

% adapted from https://tex.stackexchange.com/a/118450
\providecommand{\figcomputationvscommunicationregressionmacro}[4]{
\def\metric{#1}\def\metricslug{#2}\def\xvarslug{#3}\def\yvarprefix{#4}\input{fig/computation-vs-communication/regression/_figcomputationvscommunicationregressiontemplate.tex}
}


\figcomputationvscommunicationregressionmacro{Delivery Clumpiness}{delivery-clumpiness}{log-compute-work}{}

%\pragmaonce
% ^adapted from https://tex.stackexchange.com/a/195173

% adapted from https://tex.stackexchange.com/a/118450
\providecommand{\figcomputationvscommunicationregressionmacro}[4]{
\def\metric{#1}\def\metricslug{#2}\def\xvarslug{#3}\def\yvarprefix{#4}\input{fig/computation-vs-communication/regression/_figcomputationvscommunicationregressiontemplate.tex}
}


\figcomputationvscommunicationregressionmacro{Update Period Inlet (ns)}{simstep-period-inlet-ns}{log-compute-work}{log-}

%\pragmaonce
% ^adapted from https://tex.stackexchange.com/a/195173

% adapted from https://tex.stackexchange.com/a/118450
\providecommand{\figcomputationvscommunicationregressionmacro}[4]{
\def\metric{#1}\def\metricslug{#2}\def\xvarslug{#3}\def\yvarprefix{#4}\input{fig/computation-vs-communication/regression/_figcomputationvscommunicationregressiontemplate.tex}
}


\figcomputationvscommunicationregressionmacro{Latency Simsteps Inlet}{latency-simsteps-inlet}{log-compute-work}{log-}

%\pragmaonce
% ^adapted from https://tex.stackexchange.com/a/195173

% adapted from https://tex.stackexchange.com/a/118450
\providecommand{\figcomputationvscommunicationregressionmacro}[4]{
\def\metric{#1}\def\metricslug{#2}\def\xvarslug{#3}\def\yvarprefix{#4}\input{fig/computation-vs-communication/regression/_figcomputationvscommunicationregressiontemplate.tex}
}


\figcomputationvscommunicationregressionmacro{Simstep Period Outlet (ns)}{simstep-period-outlet-ns}{log-compute-work}{log-}

%\pragmaonce
% ^adapted from https://tex.stackexchange.com/a/195173

% adapted from https://tex.stackexchange.com/a/118450
\providecommand{\figcomputationvscommunicationregressionmacro}[4]{
\def\metric{#1}\def\metricslug{#2}\def\xvarslug{#3}\def\yvarprefix{#4}\input{fig/computation-vs-communication/regression/_figcomputationvscommunicationregressiontemplate.tex}
}


\figcomputationvscommunicationregressionmacro{Delivery Failure Rate}{delivery-failure-rate}{log-compute-work}{}


\clearpage

\pragmaonce
% ^adapted from https://tex.stackexchange.com/a/195173

% adapted from https://tex.stackexchange.com/a/118450
\newcommand{\figintranodevsinternodedistributionmacro}[2]{
\def\metric{#1}\def\metricslug{#2}\input{fig/intranode-vs-internode/distribution/_figintranodevsinternodedistributiontemplate.tex}
}


\figintranodevsinternodedistributionmacro{Latency Walltime Inlet (ns)}{latency-walltime-inlet-ns}

\pragmaonce
% ^adapted from https://tex.stackexchange.com/a/195173

% adapted from https://tex.stackexchange.com/a/118450
\newcommand{\figintranodevsinternodedistributionmacro}[2]{
\def\metric{#1}\def\metricslug{#2}\input{fig/intranode-vs-internode/distribution/_figintranodevsinternodedistributiontemplate.tex}
}


\figintranodevsinternodedistributionmacro{Latency Simsteps Outlet}{latency-simsteps-outlet}

\pragmaonce
% ^adapted from https://tex.stackexchange.com/a/195173

% adapted from https://tex.stackexchange.com/a/118450
\newcommand{\figintranodevsinternodedistributionmacro}[2]{
\def\metric{#1}\def\metricslug{#2}\input{fig/intranode-vs-internode/distribution/_figintranodevsinternodedistributiontemplate.tex}
}


\figintranodevsinternodedistributionmacro{Latency Walltime Outlet (ns)}{latency-walltime-outlet-ns}

\pragmaonce
% ^adapted from https://tex.stackexchange.com/a/195173

% adapted from https://tex.stackexchange.com/a/118450
\newcommand{\figintranodevsinternodedistributionmacro}[2]{
\def\metric{#1}\def\metricslug{#2}\input{fig/intranode-vs-internode/distribution/_figintranodevsinternodedistributiontemplate.tex}
}


\figintranodevsinternodedistributionmacro{Delivery Bunching}{delivery-clumpiness}

\pragmaonce
% ^adapted from https://tex.stackexchange.com/a/195173

% adapted from https://tex.stackexchange.com/a/118450
\newcommand{\figintranodevsinternodedistributionmacro}[2]{
\def\metric{#1}\def\metricslug{#2}\input{fig/intranode-vs-internode/distribution/_figintranodevsinternodedistributiontemplate.tex}
}


\figintranodevsinternodedistributionmacro{Simstep Period Inlet (ns)}{simstep-period-inlet-ns}

\pragmaonce
% ^adapted from https://tex.stackexchange.com/a/195173

% adapted from https://tex.stackexchange.com/a/118450
\newcommand{\figintranodevsinternodedistributionmacro}[2]{
\def\metric{#1}\def\metricslug{#2}\input{fig/intranode-vs-internode/distribution/_figintranodevsinternodedistributiontemplate.tex}
}


\figintranodevsinternodedistributionmacro{Latency Simsteps Inlet}{latency-simsteps-inlet}

\pragmaonce
% ^adapted from https://tex.stackexchange.com/a/195173

% adapted from https://tex.stackexchange.com/a/118450
\newcommand{\figintranodevsinternodedistributionmacro}[2]{
\def\metric{#1}\def\metricslug{#2}\input{fig/intranode-vs-internode/distribution/_figintranodevsinternodedistributiontemplate.tex}
}


\figintranodevsinternodedistributionmacro{Simstep Period Outlet (ns)}{simstep-period-outlet-ns}

\pragmaonce
% ^adapted from https://tex.stackexchange.com/a/195173

% adapted from https://tex.stackexchange.com/a/118450
\newcommand{\figintranodevsinternodedistributionmacro}[2]{
\def\metric{#1}\def\metricslug{#2}\input{fig/intranode-vs-internode/distribution/_figintranodevsinternodedistributiontemplate.tex}
}


\figintranodevsinternodedistributionmacro{Delivery Failure Rate}{delivery-failure-rate}


\pragmaonce
% ^adapted from https://tex.stackexchange.com/a/195173

% adapted from https://tex.stackexchange.com/a/118450
\newcommand{\figintranodevsinternoderegressionmacro}[4]{
\def\metric{#1}\def\metricslug{#2}\def\xvarslug{#3}\def\yvarprefix{#4}\input{fig/intranode-vs-internode/regression/_figintranodevsinternoderegressiontemplate.tex}
}


\figintranodevsinternoderegressionmacro{Latency Walltime Inlet (ns)}{latency-walltime-inlet-ns}{0-intranode-1-internode}{}

\pragmaonce
% ^adapted from https://tex.stackexchange.com/a/195173

% adapted from https://tex.stackexchange.com/a/118450
\newcommand{\figintranodevsinternoderegressionmacro}[4]{
\def\metric{#1}\def\metricslug{#2}\def\xvarslug{#3}\def\yvarprefix{#4}\input{fig/intranode-vs-internode/regression/_figintranodevsinternoderegressiontemplate.tex}
}


\figintranodevsinternoderegressionmacro{Latency Simsteps Outlet}{latency-simsteps-outlet}{0-intranode-1-internode}{}

\pragmaonce
% ^adapted from https://tex.stackexchange.com/a/195173

% adapted from https://tex.stackexchange.com/a/118450
\newcommand{\figintranodevsinternoderegressionmacro}[4]{
\def\metric{#1}\def\metricslug{#2}\def\xvarslug{#3}\def\yvarprefix{#4}\input{fig/intranode-vs-internode/regression/_figintranodevsinternoderegressiontemplate.tex}
}


\figintranodevsinternoderegressionmacro{Latency Walltime Outlet (ns)}{latency-walltime-outlet-ns}{0-intranode-1-internode}{}

\pragmaonce
% ^adapted from https://tex.stackexchange.com/a/195173

% adapted from https://tex.stackexchange.com/a/118450
\newcommand{\figintranodevsinternoderegressionmacro}[4]{
\def\metric{#1}\def\metricslug{#2}\def\xvarslug{#3}\def\yvarprefix{#4}\input{fig/intranode-vs-internode/regression/_figintranodevsinternoderegressiontemplate.tex}
}


\figintranodevsinternoderegressionmacro{Delivery Bunching}{delivery-clumpiness}{0-intranode-1-internode}{}

\pragmaonce
% ^adapted from https://tex.stackexchange.com/a/195173

% adapted from https://tex.stackexchange.com/a/118450
\newcommand{\figintranodevsinternoderegressionmacro}[4]{
\def\metric{#1}\def\metricslug{#2}\def\xvarslug{#3}\def\yvarprefix{#4}\input{fig/intranode-vs-internode/regression/_figintranodevsinternoderegressiontemplate.tex}
}


\figintranodevsinternoderegressionmacro{Update Period Inlet (ns)}{simstep-period-inlet-ns}{0-intranode-1-internode}{}

\pragmaonce
% ^adapted from https://tex.stackexchange.com/a/195173

% adapted from https://tex.stackexchange.com/a/118450
\newcommand{\figintranodevsinternoderegressionmacro}[4]{
\def\metric{#1}\def\metricslug{#2}\def\xvarslug{#3}\def\yvarprefix{#4}\input{fig/intranode-vs-internode/regression/_figintranodevsinternoderegressiontemplate.tex}
}


\figintranodevsinternoderegressionmacro{Latency Updates Inlet}{latency-simsteps-inlet}{0-intranode-1-internode}{}

\pragmaonce
% ^adapted from https://tex.stackexchange.com/a/195173

% adapted from https://tex.stackexchange.com/a/118450
\newcommand{\figintranodevsinternoderegressionmacro}[4]{
\def\metric{#1}\def\metricslug{#2}\def\xvarslug{#3}\def\yvarprefix{#4}\input{fig/intranode-vs-internode/regression/_figintranodevsinternoderegressiontemplate.tex}
}


\figintranodevsinternoderegressionmacro{Update Period Outlet (ns)}{simstep-period-outlet-ns}{0-intranode-1-internode}{}

\pragmaonce
% ^adapted from https://tex.stackexchange.com/a/195173

% adapted from https://tex.stackexchange.com/a/118450
\newcommand{\figintranodevsinternoderegressionmacro}[4]{
\def\metric{#1}\def\metricslug{#2}\def\xvarslug{#3}\def\yvarprefix{#4}\input{fig/intranode-vs-internode/regression/_figintranodevsinternoderegressiontemplate.tex}
}


\figintranodevsinternoderegressionmacro{Delivery Failure Rate}{delivery-failure-rate}{0-intranode-1-internode}{}


\clearpage

\pragmaonce
% ^adapted from https://tex.stackexchange.com/a/195173

% adapted from https://tex.stackexchange.com/a/118450
\newcommand{\figmultithreadingvsmultiprocessingdistributionmacro}[2]{
\def\metric{#1}\def\metricslug{#2}\input{fig/multithreading-vs-multiprocessing/distribution/_figmultithreadingvsmultiprocessingdistributiontemplate.tex}
}


\figmultithreadingvsmultiprocessingdistributionmacro{Latency Walltime Inlet (ns)}{latency-walltime-inlet-ns}

\pragmaonce
% ^adapted from https://tex.stackexchange.com/a/195173

% adapted from https://tex.stackexchange.com/a/118450
\newcommand{\figmultithreadingvsmultiprocessingdistributionmacro}[2]{
\def\metric{#1}\def\metricslug{#2}\input{fig/multithreading-vs-multiprocessing/distribution/_figmultithreadingvsmultiprocessingdistributiontemplate.tex}
}


\figmultithreadingvsmultiprocessingdistributionmacro{Latency Simsteps Outlet}{latency-simsteps-outlet}

\pragmaonce
% ^adapted from https://tex.stackexchange.com/a/195173

% adapted from https://tex.stackexchange.com/a/118450
\newcommand{\figmultithreadingvsmultiprocessingdistributionmacro}[2]{
\def\metric{#1}\def\metricslug{#2}\input{fig/multithreading-vs-multiprocessing/distribution/_figmultithreadingvsmultiprocessingdistributiontemplate.tex}
}


\figmultithreadingvsmultiprocessingdistributionmacro{Latency Walltime Outlet (ns)}{latency-walltime-outlet-ns}

\pragmaonce
% ^adapted from https://tex.stackexchange.com/a/195173

% adapted from https://tex.stackexchange.com/a/118450
\newcommand{\figmultithreadingvsmultiprocessingdistributionmacro}[2]{
\def\metric{#1}\def\metricslug{#2}\input{fig/multithreading-vs-multiprocessing/distribution/_figmultithreadingvsmultiprocessingdistributiontemplate.tex}
}


\figmultithreadingvsmultiprocessingdistributionmacro{Delivery Clumpiness}{delivery-clumpiness}

\pragmaonce
% ^adapted from https://tex.stackexchange.com/a/195173

% adapted from https://tex.stackexchange.com/a/118450
\newcommand{\figmultithreadingvsmultiprocessingdistributionmacro}[2]{
\def\metric{#1}\def\metricslug{#2}\input{fig/multithreading-vs-multiprocessing/distribution/_figmultithreadingvsmultiprocessingdistributiontemplate.tex}
}


\figmultithreadingvsmultiprocessingdistributionmacro{Simstep Period Inlet (ns)}{simstep-period-inlet-ns}

\pragmaonce
% ^adapted from https://tex.stackexchange.com/a/195173

% adapted from https://tex.stackexchange.com/a/118450
\newcommand{\figmultithreadingvsmultiprocessingdistributionmacro}[2]{
\def\metric{#1}\def\metricslug{#2}\input{fig/multithreading-vs-multiprocessing/distribution/_figmultithreadingvsmultiprocessingdistributiontemplate.tex}
}


\figmultithreadingvsmultiprocessingdistributionmacro{Latency Simsteps Inlet}{latency-simsteps-inlet}

\pragmaonce
% ^adapted from https://tex.stackexchange.com/a/195173

% adapted from https://tex.stackexchange.com/a/118450
\newcommand{\figmultithreadingvsmultiprocessingdistributionmacro}[2]{
\def\metric{#1}\def\metricslug{#2}\input{fig/multithreading-vs-multiprocessing/distribution/_figmultithreadingvsmultiprocessingdistributiontemplate.tex}
}


\figmultithreadingvsmultiprocessingdistributionmacro{Simstep Period Outlet (ns)}{simstep-period-outlet-ns}

\pragmaonce
% ^adapted from https://tex.stackexchange.com/a/195173

% adapted from https://tex.stackexchange.com/a/118450
\newcommand{\figmultithreadingvsmultiprocessingdistributionmacro}[2]{
\def\metric{#1}\def\metricslug{#2}\input{fig/multithreading-vs-multiprocessing/distribution/_figmultithreadingvsmultiprocessingdistributiontemplate.tex}
}


\figmultithreadingvsmultiprocessingdistributionmacro{Delivery Failure Rate}{delivery-failure-rate}


%\pragmaonce
% ^adapted from https://tex.stackexchange.com/a/195173

% adapted from https://tex.stackexchange.com/a/118450
\providecommand{\figmultithreadingvsmultiprocessingregressionmacro}[4]{
\def\metric{#1}\def\metricslug{#2}\def\xvarslug{#3}\def\yvarprefix{#4}\input{fig/multithreading-vs-multiprocessing/regression/_figmultithreadingvsmultiprocessingregressiontemplate.tex}
}


\figmultithreadingvsmultiprocessingregressionmacro{Latency Walltime Inlet (ns)}{latency-walltime-inlet-ns}{0-multithreading-1-multiprocessing}{}

%\pragmaonce
% ^adapted from https://tex.stackexchange.com/a/195173

% adapted from https://tex.stackexchange.com/a/118450
\providecommand{\figmultithreadingvsmultiprocessingregressionmacro}[4]{
\def\metric{#1}\def\metricslug{#2}\def\xvarslug{#3}\def\yvarprefix{#4}\input{fig/multithreading-vs-multiprocessing/regression/_figmultithreadingvsmultiprocessingregressiontemplate.tex}
}


\figmultithreadingvsmultiprocessingregressionmacro{Latency Updates Outlet}{latency-simsteps-outlet}{0-multithreading-1-multiprocessing}{}

%\pragmaonce
% ^adapted from https://tex.stackexchange.com/a/195173

% adapted from https://tex.stackexchange.com/a/118450
\providecommand{\figmultithreadingvsmultiprocessingregressionmacro}[4]{
\def\metric{#1}\def\metricslug{#2}\def\xvarslug{#3}\def\yvarprefix{#4}\input{fig/multithreading-vs-multiprocessing/regression/_figmultithreadingvsmultiprocessingregressiontemplate.tex}
}


\figmultithreadingvsmultiprocessingregressionmacro{Latency Walltime Outlet (ns)}{latency-walltime-outlet-ns}{0-multithreading-1-multiprocessing}{}

%\pragmaonce
% ^adapted from https://tex.stackexchange.com/a/195173

% adapted from https://tex.stackexchange.com/a/118450
\providecommand{\figmultithreadingvsmultiprocessingregressionmacro}[4]{
\def\metric{#1}\def\metricslug{#2}\def\xvarslug{#3}\def\yvarprefix{#4}\input{fig/multithreading-vs-multiprocessing/regression/_figmultithreadingvsmultiprocessingregressiontemplate.tex}
}


\figmultithreadingvsmultiprocessingregressionmacro{Delivery Clumpiness}{delivery-clumpiness}{0-multithreading-1-multiprocessing}{}

%\pragmaonce
% ^adapted from https://tex.stackexchange.com/a/195173

% adapted from https://tex.stackexchange.com/a/118450
\providecommand{\figmultithreadingvsmultiprocessingregressionmacro}[4]{
\def\metric{#1}\def\metricslug{#2}\def\xvarslug{#3}\def\yvarprefix{#4}\input{fig/multithreading-vs-multiprocessing/regression/_figmultithreadingvsmultiprocessingregressiontemplate.tex}
}


\figmultithreadingvsmultiprocessingregressionmacro{Update Period Inlet (ns)}{simstep-period-inlet-ns}{0-multithreading-1-multiprocessing}{}

%\pragmaonce
% ^adapted from https://tex.stackexchange.com/a/195173

% adapted from https://tex.stackexchange.com/a/118450
\providecommand{\figmultithreadingvsmultiprocessingregressionmacro}[4]{
\def\metric{#1}\def\metricslug{#2}\def\xvarslug{#3}\def\yvarprefix{#4}\input{fig/multithreading-vs-multiprocessing/regression/_figmultithreadingvsmultiprocessingregressiontemplate.tex}
}


\figmultithreadingvsmultiprocessingregressionmacro{Latency Updates Inlet}{latency-simsteps-inlet}{0-multithreading-1-multiprocessing}{}

%\pragmaonce
% ^adapted from https://tex.stackexchange.com/a/195173

% adapted from https://tex.stackexchange.com/a/118450
\providecommand{\figmultithreadingvsmultiprocessingregressionmacro}[4]{
\def\metric{#1}\def\metricslug{#2}\def\xvarslug{#3}\def\yvarprefix{#4}\input{fig/multithreading-vs-multiprocessing/regression/_figmultithreadingvsmultiprocessingregressiontemplate.tex}
}


\figmultithreadingvsmultiprocessingregressionmacro{Update Period Outlet (ns)}{simstep-period-outlet-ns}{0-multithreading-1-multiprocessing}{}

%\pragmaonce
% ^adapted from https://tex.stackexchange.com/a/195173

% adapted from https://tex.stackexchange.com/a/118450
\providecommand{\figmultithreadingvsmultiprocessingregressionmacro}[4]{
\def\metric{#1}\def\metricslug{#2}\def\xvarslug{#3}\def\yvarprefix{#4}\input{fig/multithreading-vs-multiprocessing/regression/_figmultithreadingvsmultiprocessingregressiontemplate.tex}
}


\figmultithreadingvsmultiprocessingregressionmacro{Delivery Failure Rate}{delivery-failure-rate}{0-multithreading-1-multiprocessing}{}



\end{document}
