\section{Conclusion}

Benchmarks show that best-effort communication through Conduit enables significantly better computational performance under high thread and process counts.
We also demonstrated how, in the case of the graph coloring benchmark, best-effort communication can help achieve tangibly better solution quality within a fixed time constraint, as well.  
We observed the greatest relative speedup under distributed communciation-heavy workloads --- about $7.8\times$ on the graph coloring benchmark.
Distributing the computation-heavy digital evolution benchmark workload across independent nodes yielded the strongest scaling of our benchmarks, achieving at 64 processes 92\% the update-rate of single-process execution. 

In future work, we plan to further characterize the performance of the Conduit's best-effort model with respect to the digital evolution simulation, looking directly at quality of service metrics such as message latency and frequency of dropped messages.
We are also eager to investigate how Conduit's best-effort communication model scales on much larger process counts, perhaps on the order of hundreds or thousands of cores.

Development of the Conduit library stemmed from a practical need for an abstract, prepackaged interface to support our digital evolution research.
We hope that making this library available to the community can reduce domain expertise and programmability barriers to taking advantage of the best-effort communication model to efficiently leverage burgeoning parallel and distributed computing power. 