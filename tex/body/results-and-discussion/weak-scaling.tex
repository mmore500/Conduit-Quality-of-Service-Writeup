\subsection{Quality of Service: Weak Scaling}

Sections \ref{sec:multiprocess-benchmarks} and \ref{sec:multithread-benchmarks} showed how best-effort communication could improve application performance, particularly when scaling up processor count.
Performance exhibited promising properties under multiprocess scaling, with overlapping performance estimate 95\% confidence intervals for 16 and 64 processor counts on both surveyed benchmark problems.
This section aims to flesh out understanding of how scaling processor count affects a comprehensive suite of quality of service metrics, with particular interest in which, if any, aspects of quality of service degrade for larger processing pools.

To address these questions, we performed weak scaling experiments on 16, 64, and 256 processes using the graph coloring benchmark.
To broaden the survey, we tested scaling under four treatments from the Cartesian product of two variables: processors allocated per node and simulation elements assigned per processor.
For the first variable, we tested scaling on allocations with each processor hosted on an independent node and allocations where each node hosted an average of four processors.
This allowed us to examine how quality of service fared in homogeneous network conditions, where all communication between processes was inter-node, compared to heteregeneous conditions, where some inter-process communication was inter-node and some was intra-node.
For the second variable, we tested with 2'048 simulation elements (``simels'') per processor (consistent with the benchmarking experiments performed in Sections \ref{sec:multiprocess-benchmarks} and \ref{sec:multithreading-benchmarks}) and just one simulation element per processor, maximizing communication relative to computation.

\subsubsection{Simstep Period}

Supplementary Figures \ref{fig:weak-scaling-distribution-simstep-period-inlet-ns} and \ref{fig:weak-scaling-distribution-simstep-period-outlet-ns} survey the distributions of simstep periods observed within snapshot windows.
Simstep period registers around \SI{80}{\micro\second} with one simel and around \SI{200}{\micro\second} with 2'048 simels.
However, on heterogeneous allocations (4 cpus per node) this metric is more variable, spanning up to an order of magnitude.
Outlier observations range up to around 10ms with 2'048 simels and up to slightly less than 100ms inlet / 4s outlet with 1 simel.

We performed an ordinary least squares (OLS) regression to test how mean simstep period changed with processor count.
In all cases except one simel per cpu with four cpus per node, mean simstep period increased significantly with processor count from 16 to 64 to 256.
However, from 64 to 256 processors mean simstep period only increased significantly with one simel per cpu and one cpu per node.
Between 64 and 256 processes, mean simstep period actually decreased significantly for runs with 2'048 simels per cpu.
Supplementary Figures \ref{fig:weak-scaling-regression-ols-simstep-period-inlet-ns} and \ref{fig:weak-scaling-regression-ols-simstep-period-outlet-ns} visualize reported OLS regressions.
Supplementary Tables \ref{tab:weak-scaling-simstep-period-inlet-ns-regression-ols} and \ref{tab:weak-scaling-simstep-period-outlet-ns-regression-ols} provide numerical details on reported OLS regressions.

Median simstep period exhibited the same relationships with processor count, tested with quartile regression.
Supplementary Figures \ref{fig:weak-scaling-regression-quantile-simstep-period-inlet-ns} and \ref{fig:weak-scaling-regression-quantile-simstep-period-outlet-ns} visualize corresponding quartile regressions.
Supplementary Tables \ref{tab:weak-scaling-simstep-period-inlet-ns-regression-quantile} and \ref{tab:weak-scaling-simstep-period-outlet-ns-regression-quantile} report numerical details on those quartile regressions.

\subsubsection{Walltime Latency}

distribution Supplementary Figures \ref{fig:weak-scaling-distribution-latency-walltime-inlet-ns} and \ref{fig:weak-scaling-distribution-latency-walltime-outlet-ns}

ordinary least squares regression Supplementary Figures \ref{fig:weak-scaling-regression-ols-latency-walltime-inlet-ns} and \ref{fig:weak-scaling-regression-ols-latency-walltime-outlet-ns} Supplementary Tables \ref{tab:weak-scaling-latency-walltime-inlet-ns-regression-ols} and \ref{tab:weak-scaling-latency-walltime-outlet-ns-regression-ols}

quantile regression Supplementary Figures \ref{fig:weak-scaling-regression-quantile-latency-walltime-inlet-ns} and \ref{fig:weak-scaling-regression-quantile-latency-walltime-outlet-ns}
Supplementary Tables \ref{tab:weak-scaling-latency-walltime-inlet-ns-regression-quantile} \ref{tab:weak-scaling-latency-walltime-outlet-ns-regression-quantile}


\subsubsection{Simstep Latency}

distribution Supplementary Figures \ref{fig:weak-scaling-distribution-latency-simsteps-inlet} and \ref{fig:weak-scaling-distribution-latency-simsteps-outlet}

ordinary least squares regression Supplementary Figures \ref{fig:weak-scaling-regression-ols-latency-simsteps-inlet} and \ref{fig:weak-scaling-regression-ols-latency-simsteps-outlet} Supplementary Tables \ref{tab:weak-scaling-latency-simsteps-inlet-regression-ols} and \ref{tab:weak-scaling-latency-simsteps-outlet-regression-ols}

quantile regression Supplementary Figures \ref{fig:weak-scaling-regression-quantile-latency-simsteps-inlet} and \ref{fig:weak-scaling-regression-quantile-latency-simsteps-outlet}
Supplementary Tables \ref{tab:weak-scaling-latency-simsteps-inlet-regression-quantile} and \ref{tab:weak-scaling-latency-simsteps-outlet-regression-quantile}

\subsubsection{Delivery Clumpiness}

distribution Supplementary Figure \ref{fig:weak-scaling-distribution-delivery-clumpiness}

ordinary least squares regression Supplementary Figure \ref{fig:weak-scaling-regression-ols-delivery-clumpiness} Supplementary Table \ref{tab:weak-scaling-delivery-clumpiness-regression-ols}

quantile regression Supplementary Figure \ref{fig:weak-scaling-regression-quantile-delivery-clumpiness}
Supplementary Table \ref{tab:weak-scaling-delivery-clumpiness-regression-quantile}


\subsubsection{Delivery Failure Rate}

distribution Supplementary Figure \ref{fig:weak-scaling-distribution-delivery-failure-rate}

ordinary least squares regression Supplementary Figure \ref{fig:weak-scaling-regression-ols-delivery-failure-rate} Supplementary Table \ref{tab:weak-scaling-delivery-failure-rate-regression-ols}

quantile regression Supplementary Figure \ref{fig:weak-scaling-regression-quantile-delivery-failure-rate}
Supplementary Table \ref{tab:weak-scaling-delivery-failure-rate-regression-quantile}
