\section{Introduction}

The parallel and distributed processing capacity of high-performance computing clusters continues to grow rapidly and enable profound scientific and industrial innovations \cite{gagliardi2019international}.
Hardware advances afford great opportunity, but also pose a serious challenge: developing approaches to effectively harness it.
As HPC systems scale, deterministic algorithms depending on global synchronization become increasingly costly \cite{gropp2013programming,dongarra2014applied}.

Best-effort computing, where data dependencies are relaxed to reduce synchronization \cite{chakradhar2010best}, can improve scalability \cite{meng2009best}.
Evolutionary algorithms typically perform perform a search or optimization with many acceptable results using pseudo-stochastic methods.
Algorithms with these properties can often tolerate significant perturbations, including relaxation of synchronization requirements in the underlying algorithm.
For example, island model genetic algorithms have been shown to perform well with asynchronous migration \cite{izzo2009parallel}.
While not the focus of this paper, the original interest that motivated development of the Conduit library stems from work exploring open-ended evolution.
Researchers in this domain study long-term dynamics of evolutionary systems in order to understand factors that affect these systems' potential to generate ongoing novelty \cite{taylor2016open}.
Some recent evidence suggests that the generative potential of systems devised to study open-ended evolution is --- at least in part --- meaningfully constrained by available compute resources \cite{channon2019maximum}.
Such observations raise the question of how to design parallel and distributed open-ended evolution systems.

The concept of indefinite scalability was developed to describe constraints distributed systems would face at the asymptote of technological (and even physical) constraints.
Indefinite scalability theory posits that such systems would necessitate
\begin{itemize}
    \item asynchronous operation,
    \item decentralized networking,
    \item interchangeable components (i.e., no global identifiers),
    \item and graceful degradation under hardware failure \cite{ackley2011pursue}.
\end{itemize}
Although bespoke experimental hardware such as Illuminato X machina has been developed to demonstrate aspects of the indefinite scalability paradigm \cite{ackley2011homeostatic}, practical contemporary work scaling up evolving systems necessitates working with commercially-available hardware.
However, existing distributed computing frameworks for contemporary high-performance computing hardware do not explicitly expose a convenient best-effort interface.
In our current work, we focus on asynchronous operation and decentralization.
We leave system robustness and resistance to hardware degradation to future work.

The Message Passing Interface (MPI) standard \cite{gropp1996high} represents the mainstay for high-performance computing applications.
This standard exposes communication primitives directly to the end user.
MPI's nonblocking communication primitives, in particular, are sufficient to program distributed computations with relaxed synchronization requirements.
Although its explicit, imperative nature enables precise control over execution, it also poses significant expense in terms of programability.
This cost manifests in terms of programmer productivity, domain knowledge requirements, software quality, and difficulty tuning for performance due to program brittleness \cite{gu2019comparative, tang2014mpi}.

In response to programmability concerns, many frameworks have arisen to offer useful parallel and distributed programming abstractions.
Task-based frameworks such as Charm++ \cite{kale1993charm++}, Legion \cite{bauer2012legion}, Cilk \cite{blumofe1996cilk}, and Threading Building Blocks (TBB) \cite{reinders2007intel} describe the dependency relationships among computational tasks and associated data and relies on an associated runtime to automatically schedule and manage execution.
These frameworks assume a deterministic relationship between tasks.
In a similar vein, programming languages and extensions like Unified Parallel C (UPC) \cite{el2006upc} and Chapel \cite{chamberlain2007parallel} rely on programmers to direct execution, but equips them with powerful abstractions, such as global shared memory.
However, Chapel's memory model explicitly forbids data races and UPC ultimately relies on a barrier model for data transfer.
