\section{Conclusion}

The fundamental motivation for best-effort communication is efficient scalability.
Our results confirm that best-effort communication can fulfill on this goal.

We found that the best-effort approach significantly increases performance at high CPU count.
This finding was consistent across the communication-intensive graph coloring benchmark and the computation-intensive digital evolution benchmark.
The computation-heavy digital evolution benchmark yielded the strongest scaling efficiency, achieving at 64 processes 92\% the update-rate of single-process execution.
We observed the greatest relative speedup under distributed communication-heavy workloads --- about $7.8\times$ on the graph coloring benchmark.
In the case of the graph coloring benchmark, we found that best-effort communication can help achieve tangibly better solution quality within a fixed time constraint.

Because real-time volatility affects the outcome of computation under the best-effort model, raw execution speed performance does not suffice to fully understand the consequences of the best-effort communication model.
In order to characterize the real-time dynamics under the best-effort model, we designed and measured a suite of quality of service metrics: simstep period, simstep latency, wall-time latency, delivery failure rate, and delivery clumpiness.

We performed several experiments to validate and characterize these metrics.
Comparing quality of service between multithreading and multiprocessing, we found that multithreading had lower runtime overhead cost but that multiprocessing reduced delivery erraticity, curbing especially extreme poor quality of service outlier events.
We found better quality of service, especially with respect to latency, for processes occupying the same node.
Finally, varying the ratio of computational work to communication, we found lower communication intensity associated with less volatile quality of service.

In order for best-effort communication to succeed in facilitating scale-up, median quality of service must stabilize with increasing CPU count.
Put another way, best-effort communication cannot succeed at scale if communication quality tends toward complete degradation.
In Section \ref{sec:weak-scaling}, we used weak scaling experiments to test the effect of scale-up on quality of service at 8, 64, and 256 processes.
Under a lower communication-intensivity task parameterization, we found that all median quality of service metrics were stable when scaling from 64 to 256 process.
Under maximal communication intensivity, we found in one case that median simstep period degraded from around \SI{80}{\micro\second} to around \SI{85}{\micro\second}.
In another case, median message delivery failure rate increased from around 7\% to around 9\%.
Such minor --- and, in most cases, nil --- degradation in median quality of service despite maximal communication intensivity bodes well for the viability of best-effort communication at scale.

Resilience is a second major motivating factor for best-effort computing.
In another promising result, we found that the presence of an apparently faulty compute node did not degrade median performance or quality of service.
Despite extreme quality of service degradation measured among that node and its clique, collective performance and quality of service remained steady.
In effect, the best-effort approach successfully decoupled global performance from the worst performer.
Such so-called ``straggler effects'' plague traditional approaches to large-scale high-performance computing \citep{aktacs2019straggler}, so avoiding them is a major boon.

Development of the Conduit library stemmed from a practical need for an abstract, prepackaged best-effort commmunication interface to support our digital evolution research.
Because real-time effects are fundamentally application-dependent and arise without any explicit in-program specification (and therefore may be unanticipated) it is important to be able to perform such quality of service profiling case-by-case in applications of best-effort communication.
The instrumentation used in these experiments is written as wrappers around the library's \texttt{Inlet} and \texttt{Outlet} classes that may be enabled via compile-time configuration switch.
This makes data generation for quality of service analysis trivial to perform in any system built with the Conduit library.
We hope that making this library and quality of service metrics available to the community can reduce domain expertise and programmability barriers to taking advantage of the best-effort communication model to efficiently leverage burgeoning parallel and distributed computing power.

In future work, it may be of interest to design systems that monitor and proactively react to real-time quality of service conditions.
For example, imposing a variable cost for cell-cell messaging to agents based on traffic levels or increasing per-update resource generation rates for agents on slow-running nodes.
We are eager to investigate how Conduit's best-effort communication model scales on much larger process counts on the order of thousands of cores.
