\section{Conclusion}

Benchmarks in Sections \ref{sec:multiprocess-benchmarks} and \ref{sec:multithread-benchmarks} show that best-effort communication through Conduit enables significantly better computational performance under high thread and process counts.
We also demonstrated how, in the case of the graph coloring benchmark, best-effort communication can help achieve tangibly better solution quality within a fixed time constraint, as well.
We observed the greatest relative speedup under distributed communciation-heavy workloads --- about $7.8\times$ on the graph coloring benchmark.
Distributing the computation-heavy digital evolution benchmark workload across independent nodes yielded the strongest scaling of our benchmarks, achieving at 64 processes 92\% the update-rate of single-process execution.

Weak scaling analysis of quality of service metrics in Section \ref{sec:weak-scaling} revealed that median quality of service remains stable under increasing process counts, with the sole exception of delivery failure rate under some circumstances.
%TODO recap the quality of service descriptive experiments
%TODO add more detailed recap of weak scaling results
%TODO tell the implications of weak scaling results

Development of the Conduit library stemmed from a practical need for an abstract, prepackaged interface to support our digital evolution research.
Because real-time effects are fundamentally application-dependent and arise without any explicit in-program specification (and therefore may be unanticipated) it is important to be able to perform such profiling case-by-case in applications of best-effort communication.
The instrumentation used in these experiments is written as wrappers around Inlets and Outlets that may be enabled via compile-time configuration switch.
This makes data generation for quality of service analysis trivial to perform in any system built with the Conduit library.
We hope that making this library and quality of service metrics available to the community can reduce domain expertise and programmability barriers to taking advantage of the best-effort communication model to efficiently leverage burgeoning parallel and distributed computing power.

In future work, it may be of interest to design systems that monitor and proactively react to real-time quality of service conditions.
For example, imposing a variable cost for cell-cell messaging to agents based on traffic levels or increasing per-update resource generation for agents on slow-running nodes.
We are eager to investigate how Conduit's best-effort communication model scales on much larger process counts, perhaps on the order of hundreds or thousands of cores.
