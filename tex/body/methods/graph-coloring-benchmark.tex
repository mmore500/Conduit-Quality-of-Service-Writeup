\subsection{ Graph Coloring Benchmark } \label{sec:graph_coloring_benchmark}

The graph coloring benchmark employs a graph coloring algorithm designed for distributed WLAN channel selection \citep{leith2012wlan}.
In this algorithm, nodes begin by randomly choosing a color.
Each computational update, nodes test for any neighbor with the same color.
If and only if a conflicting neighbor is detected, nodes randomly select another color.
The probability of selecting each possible color is stored in array associated with each node.
Before selecting a new color, the stored probability of selecting the current (conflicting) color is decreased by a multiplicative factor $b$.
We used $b=0.1$, as suggested by Leith et al.
Likewise, the stored probability of selecting all others is increased by a multiplicative factor.
Regardless of whether their color changed, nodes transmit their current color to their neighbor.

Our benchmarks focus on weak scalability, using a fixed problem size of 2048 graph nodes per thread or process.
These nodes were arranged in a two-dimensional grid topology where each node had three possible colors and four neighbors.
We implement the algorithm with a single Conduit communication layer carrying graph color as an unsigned integer.
We used Conduit's built-in pooling feature to consolidate color information into a single MPI message between pairs of communicating processes each update.
We performed five replicates, each with a five second simulation runtime.
Solution error was measured as the number of graph color conflicts remaining at the end of the benchmark.
