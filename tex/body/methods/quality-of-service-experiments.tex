\subsection{Quality of Service Experiments} \label{sec:quality-of-service-experiments}

real-time
In order to

Inlet vs Outlet

We measure these metrics during one-second windows occurring once a minute.
Measurements are sampled independently within each window on each processing element.
For each quality of service statistic we estimate mean --- which captures effects of extreme-magnitude outliers --- and median --- which more closely represents typicalness --- across these window samples.


1 second live measurements from another thread
To aid consistent interpretation, all metrics are constructed such that mininimization represents desirable high quality of service.



1 second live measurements from another thread

\subsubsection{Latency}

Simsteps and walltime.

To measure simstep latency, we round counted trip messages.
Messages between elements were tagged with a counter that.
Between inlet and outlet node pairs

\begin{itemize}
  \item latency (simsteps)
    - \verb|(df['Num Puts Attempted'] - 1) / df['Num Round Trip Touches Inlet']|
  \item latency (walltime)
    - \verb|df['Latency Simsteps Inlet'] * df['Simstep Period Inlet (s)']|
  \item delivery clumpiness
    - \verb|1.0 - df_snapshot_diffs['Num Pulls That Were Laden Immediately'] / df_snapshot_diffs[['Net Flux Through Duct', 'Num Pulls Attempted']].min(axis=1)|
  \item delivery failure rate
    - \verb|message sends attempted / messages delivered|
  \item simstep period (walltime)
    - \verb|df_snapshot_diffs['Inlet-Nanoseconds Elapsed'] / df_snapshot_diffs['Num Puts Attempted']|
\end{itemize}

How do we calculate each of these?

Inlet vs Outlet


We measure these metrics during one-second windows occurring once a minute.
Measurements are sampled independently within each window on each processing element.
For each quality of service statistic we estimate mean --- which captures effects of extreme-magnitude outliers --- and median --- which more closely represents typicalness --- across these window samples.
