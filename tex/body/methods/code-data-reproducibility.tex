\subsection{Code, Data, and Reproducibility}

\subsubsection{Benchmarking Experiments} \label{sec:methods-code-data-reproducibility-benchmarking-experiments}

Benchmarking experiments were performed on Michigan State University's High Performance Computing Center, a cluster of hundreds of heterogeneous x86 nodes linked with InfiniBand interconnects.
For multithread experiments, benchmarks for each thread count were collected from the same node.
For multiprocess experiments, each processes was assigned to a distinct node in order to ensure results were representative of performance in a distributed context.
All multiprocess benchmarks were recorded from the same collection of nodes.
Hostnames are recorded for each benchmark data point.
For an exact accounting of hardware architectures used, these hostnames can be crossreferenced with a table included with the data that summarizes the cluster's node configurations.

Code for the distributed graph coloring benchmark is available at \url{https://github.com/mmore500/conduit} under \\ \texttt{demos/channel\_selection}.
Code for the digital evolution simulation benchmark is available at \url{https://github.com/mmore500/dishtiny}.
Exact versions of software used are recorded with each benchmark data point.
Data is available via the Open Science Framework at \url{https://osf.io/7jkgp/} and \url{https://osf.io/72k5n} \citep{foster2017open}.
A live, in-browser notebook for all reported statistics and data visualizations and is available via Binder at \url{https://mybinder.org/v2/gh/mmore500/conduit/binder?filepath=binder%2Fdate%3D2021%2Bproject%3D72k5n} \citep{jupyter2018binder}.

\subsubsection{Quality of Service Experiments}

Quality of service experiments were performed on Quality of service experiments were carried out on Michigan State University's High Performance Computing Center lac cluster, consisting of 28-core Intel(R) Xeon(R) CPU E5-2680 v4 \@ 2.40GHz nodes.
All statistical comparisons are performed between observations from the same job allocation.
(Except in the case where intranode and internode configurations were compared, where experiments were performed on separate allocations using comparable nodes on the same cluster.)

Benchmarking experiments described in Section \ref{sec:methods-code-data-reproducibility-benchmarking-experiments} used a send/receive buffer size of 2.
However, due to the high communication intensity of the graph coloring problem with just one simulation element per CPU, quality of service experiments required a larger buffer size of 64 to maintain runtime stability.
In early work developing the Conduit library, we discovered that real-time messaging channels can enter a destabilizing positive feedback spiral when incoming messages take longer to handle (e.g., skip past or read) than sending messages.
Under such conditions, when a process exchanging messages from a partner process experiences a delay it sends fewer messages to that partner process.
Due to fewer incoming messages, the partner the partner process can update more rapidly, increasing incoming message load on the delayed process.
This effect can snowball the partnership intended for even, two-way message exchange into effectively a unilateral producer-consumer relationship where (potentially unbounded) work piles up on the consumer.
To interrupt such a scenario, we use the bulk message pull call \verb|MPI_Testsome| to ensure fast message consumption under backlogged conditions.
So, receiver workload remains closer to constant under high traffic situations (instead of having to pull messages down one-by-one).
Larger receive buffer size, as configured for the quality of service experiments, increases the effectiveness of the bulk message consumption countermeasure.

Code for the distributed graph coloring benchmark is available at \url{https://github.com/mmore500/conduit} under \\ \texttt{demos/channel\_selection}.
Exact versions of software used are recorded with each benchmark data point.
Data is available via the Open Science Framework at \url{https://osf.io/72k5n/} \citep{foster2017open}.
A live, in-browser notebook for all reported statistics and data visualizations is available via Binder at \url{https://mybinder.org/v2/gh/mmore500/conduit/binder?filepath=binder%2Fdate%3D2021%2Bproject%3D72k5n} \citep{jupyter2018binder}.
